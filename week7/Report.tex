\documentclass[11pt]{article}

    \usepackage{listings}

    \usepackage[breakable]{tcolorbox}
    \usepackage{parskip} % Stop auto-indenting (to mimic markdown behaviour)

    \usepackage{iftex}
    \ifPDFTeX
    	\usepackage[T1]{fontenc}
    	\usepackage{mathpazo}
    \else
    	\usepackage{fontspec}
    \fi

    \usepackage{graphicx} %package to manage images
    \graphicspath{ {./images/} }
    \usepackage{wrapfig}

    % Basic figure setup, for now with no caption control since it's done
    % automatically by Pandoc (which extracts ![](path) syntax from Markdown).
    \usepackage{graphicx}
    % Maintain compatibility with old templates. Remove in nbconvert 6.0
    \let\Oldincludegraphics\includegraphics
    % Ensure that by default, figures have no caption (until we provide a
    % proper Figure object with a Caption API and a way to capture that
    % in the conversion process - todo).
    \usepackage{caption}
    \DeclareCaptionFormat{nocaption}{}
    \captionsetup{format=nocaption,aboveskip=0pt,belowskip=0pt}

    \usepackage[Export]{adjustbox} % Used to constrain images to a maximum size
    \adjustboxset{max size={0.9\linewidth}{0.9\paperheight}}
    \usepackage{float}
    \floatplacement{figure}{H} % forces figures to be placed at the correct location
    \usepackage{xcolor} % Allow colors to be defined
    \usepackage{enumerate} % Needed for markdown enumerations to work
    \usepackage{geometry} % Used to adjust the document margins
    \usepackage{amsmath} % Equations
    \usepackage{amssymb} % Equations
    \usepackage{textcomp} % defines textquotesingle
    % Hack from http://tex.stackexchange.com/a/47451/13684:
    \AtBeginDocument{%
        \def\PYZsq{\textquotesingle}% Upright quotes in Pygmentized code
    }
    \usepackage{upquote} % Upright quotes for verbatim code
    \usepackage{eurosym} % defines \euro
    \usepackage[mathletters]{ucs} % Extended unicode (utf-8) support
    \usepackage{fancyvrb} % verbatim replacement that allows latex
    \usepackage{grffile} % extends the file name processing of package graphics
                         % to support a larger range
    \makeatletter % fix for grffile with XeLaTeX
    \def\Gread@@xetex#1{%
      \IfFileExists{"\Gin@base".bb}%
      {\Gread@eps{\Gin@base.bb}}%
      {\Gread@@xetex@aux#1}%
    }
    \makeatother

    % The hyperref package gives us a pdf with properly built
    % internal navigation ('pdf bookmarks' for the table of contents,
    % internal cross-reference links, web links for URLs, etc.)
    \usepackage{hyperref}
    % The default LaTeX title has an obnoxious amount of whitespace. By default,
    % titling removes some of it. It also provides customization options.
    \usepackage{titling}
    \usepackage{longtable} % longtable support required by pandoc >1.10
    \usepackage{booktabs}  % table support for pandoc > 1.12.2
    \usepackage[inline]{enumitem} % IRkernel/repr support (it uses the enumerate* environment)
    \usepackage[normalem]{ulem} % ulem is needed to support strikethroughs (\sout)
                                % normalem makes italics be italics, not underlines
    \usepackage{mathrsfs}



    % Colors for the hyperref package
    \definecolor{urlcolor}{rgb}{0,.145,.698}
    \definecolor{linkcolor}{rgb}{.71,0.21,0.01}
    \definecolor{citecolor}{rgb}{.12,.54,.11}

    % ANSI colors
    \definecolor{ansi-black}{HTML}{3E424D}
    \definecolor{ansi-black-intense}{HTML}{282C36}
    \definecolor{ansi-red}{HTML}{E75C58}
    \definecolor{ansi-red-intense}{HTML}{B22B31}
    \definecolor{ansi-green}{HTML}{00A250}
    \definecolor{ansi-green-intense}{HTML}{007427}
    \definecolor{ansi-yellow}{HTML}{DDB62B}
    \definecolor{ansi-yellow-intense}{HTML}{B27D12}
    \definecolor{ansi-blue}{HTML}{208FFB}
    \definecolor{ansi-blue-intense}{HTML}{0065CA}
    \definecolor{ansi-magenta}{HTML}{D160C4}
    \definecolor{ansi-magenta-intense}{HTML}{A03196}
    \definecolor{ansi-cyan}{HTML}{60C6C8}
    \definecolor{ansi-cyan-intense}{HTML}{258F8F}
    \definecolor{ansi-white}{HTML}{C5C1B4}
    \definecolor{ansi-white-intense}{HTML}{A1A6B2}
    \definecolor{ansi-default-inverse-fg}{HTML}{FFFFFF}
    \definecolor{ansi-default-inverse-bg}{HTML}{000000}

    % commands and environments needed by pandoc snippets
    % extracted from the output of `pandoc -s`
    \providecommand{\tightlist}{%
      \setlength{\itemsep}{0pt}\setlength{\parskip}{0pt}}
    \DefineVerbatimEnvironment{Highlighting}{Verbatim}{commandchars=\\\{\}}
    % Add ',fontsize=\small' for more characters per line
    \newenvironment{Shaded}{}{}
    \newcommand{\KeywordTok}[1]{\textcolor[rgb]{0.00,0.44,0.13}{\textbf{{#1}}}}
    \newcommand{\DataTypeTok}[1]{\textcolor[rgb]{0.56,0.13,0.00}{{#1}}}
    \newcommand{\DecValTok}[1]{\textcolor[rgb]{0.25,0.63,0.44}{{#1}}}
    \newcommand{\BaseNTok}[1]{\textcolor[rgb]{0.25,0.63,0.44}{{#1}}}
    \newcommand{\FloatTok}[1]{\textcolor[rgb]{0.25,0.63,0.44}{{#1}}}
    \newcommand{\CharTok}[1]{\textcolor[rgb]{0.25,0.44,0.63}{{#1}}}
    \newcommand{\StringTok}[1]{\textcolor[rgb]{0.25,0.44,0.63}{{#1}}}
    \newcommand{\CommentTok}[1]{\textcolor[rgb]{0.38,0.63,0.69}{\textit{{#1}}}}
    \newcommand{\OtherTok}[1]{\textcolor[rgb]{0.00,0.44,0.13}{{#1}}}
    \newcommand{\AlertTok}[1]{\textcolor[rgb]{1.00,0.00,0.00}{\textbf{{#1}}}}
    \newcommand{\FunctionTok}[1]{\textcolor[rgb]{0.02,0.16,0.49}{{#1}}}
    \newcommand{\RegionMarkerTok}[1]{{#1}}
    \newcommand{\ErrorTok}[1]{\textcolor[rgb]{1.00,0.00,0.00}{\textbf{{#1}}}}
    \newcommand{\NormalTok}[1]{{#1}}

    % Additional commands for more recent versions of Pandoc
    \newcommand{\ConstantTok}[1]{\textcolor[rgb]{0.53,0.00,0.00}{{#1}}}
    \newcommand{\SpecialCharTok}[1]{\textcolor[rgb]{0.25,0.44,0.63}{{#1}}}
    \newcommand{\VerbatimStringTok}[1]{\textcolor[rgb]{0.25,0.44,0.63}{{#1}}}
    \newcommand{\SpecialStringTok}[1]{\textcolor[rgb]{0.73,0.40,0.53}{{#1}}}
    \newcommand{\ImportTok}[1]{{#1}}
    \newcommand{\DocumentationTok}[1]{\textcolor[rgb]{0.73,0.13,0.13}{\textit{{#1}}}}
    \newcommand{\AnnotationTok}[1]{\textcolor[rgb]{0.38,0.63,0.69}{\textbf{\textit{{#1}}}}}
    \newcommand{\CommentVarTok}[1]{\textcolor[rgb]{0.38,0.63,0.69}{\textbf{\textit{{#1}}}}}
    \newcommand{\VariableTok}[1]{\textcolor[rgb]{0.10,0.09,0.49}{{#1}}}
    \newcommand{\ControlFlowTok}[1]{\textcolor[rgb]{0.00,0.44,0.13}{\textbf{{#1}}}}
    \newcommand{\OperatorTok}[1]{\textcolor[rgb]{0.40,0.40,0.40}{{#1}}}
    \newcommand{\BuiltInTok}[1]{{#1}}
    \newcommand{\ExtensionTok}[1]{{#1}}
    \newcommand{\PreprocessorTok}[1]{\textcolor[rgb]{0.74,0.48,0.00}{{#1}}}
    \newcommand{\AttributeTok}[1]{\textcolor[rgb]{0.49,0.56,0.16}{{#1}}}
    \newcommand{\InformationTok}[1]{\textcolor[rgb]{0.38,0.63,0.69}{\textbf{\textit{{#1}}}}}
    \newcommand{\WarningTok}[1]{\textcolor[rgb]{0.38,0.63,0.69}{\textbf{\textit{{#1}}}}}


    % Define a nice break command that doesn't care if a line doesn't already
    % exist.
    \def\br{\hspace*{\fill} \\* }
    % Math Jax compatibility definitions
    \def\gt{>}
    \def\lt{<}
    \let\Oldtex\TeX
    \let\Oldlatex\LaTeX
    \renewcommand{\TeX}{\textrm{\Oldtex}}
    \renewcommand{\LaTeX}{\textrm{\Oldlatex}}
    % Document parameters
    % Document title
    \title{2-factor-Hull-White\_model\_calibration}





% Pygments definitions
\makeatletter
\def\PY@reset{\let\PY@it=\relax \let\PY@bf=\relax%
    \let\PY@ul=\relax \let\PY@tc=\relax%
    \let\PY@bc=\relax \let\PY@ff=\relax}
\def\PY@tok#1{\csname PY@tok@#1\endcsname}
\def\PY@toks#1+{\ifx\relax#1\empty\else%
    \PY@tok{#1}\expandafter\PY@toks\fi}
\def\PY@do#1{\PY@bc{\PY@tc{\PY@ul{%
    \PY@it{\PY@bf{\PY@ff{#1}}}}}}}
\def\PY#1#2{\PY@reset\PY@toks#1+\relax+\PY@do{#2}}

\expandafter\def\csname PY@tok@w\endcsname{\def\PY@tc##1{\textcolor[rgb]{0.73,0.73,0.73}{##1}}}
\expandafter\def\csname PY@tok@c\endcsname{\let\PY@it=\textit\def\PY@tc##1{\textcolor[rgb]{0.25,0.50,0.50}{##1}}}
\expandafter\def\csname PY@tok@cp\endcsname{\def\PY@tc##1{\textcolor[rgb]{0.74,0.48,0.00}{##1}}}
\expandafter\def\csname PY@tok@k\endcsname{\let\PY@bf=\textbf\def\PY@tc##1{\textcolor[rgb]{0.00,0.50,0.00}{##1}}}
\expandafter\def\csname PY@tok@kp\endcsname{\def\PY@tc##1{\textcolor[rgb]{0.00,0.50,0.00}{##1}}}
\expandafter\def\csname PY@tok@kt\endcsname{\def\PY@tc##1{\textcolor[rgb]{0.69,0.00,0.25}{##1}}}
\expandafter\def\csname PY@tok@o\endcsname{\def\PY@tc##1{\textcolor[rgb]{0.40,0.40,0.40}{##1}}}
\expandafter\def\csname PY@tok@ow\endcsname{\let\PY@bf=\textbf\def\PY@tc##1{\textcolor[rgb]{0.67,0.13,1.00}{##1}}}
\expandafter\def\csname PY@tok@nb\endcsname{\def\PY@tc##1{\textcolor[rgb]{0.00,0.50,0.00}{##1}}}
\expandafter\def\csname PY@tok@nf\endcsname{\def\PY@tc##1{\textcolor[rgb]{0.00,0.00,1.00}{##1}}}
\expandafter\def\csname PY@tok@nc\endcsname{\let\PY@bf=\textbf\def\PY@tc##1{\textcolor[rgb]{0.00,0.00,1.00}{##1}}}
\expandafter\def\csname PY@tok@nn\endcsname{\let\PY@bf=\textbf\def\PY@tc##1{\textcolor[rgb]{0.00,0.00,1.00}{##1}}}
\expandafter\def\csname PY@tok@ne\endcsname{\let\PY@bf=\textbf\def\PY@tc##1{\textcolor[rgb]{0.82,0.25,0.23}{##1}}}
\expandafter\def\csname PY@tok@nv\endcsname{\def\PY@tc##1{\textcolor[rgb]{0.10,0.09,0.49}{##1}}}
\expandafter\def\csname PY@tok@no\endcsname{\def\PY@tc##1{\textcolor[rgb]{0.53,0.00,0.00}{##1}}}
\expandafter\def\csname PY@tok@nl\endcsname{\def\PY@tc##1{\textcolor[rgb]{0.63,0.63,0.00}{##1}}}
\expandafter\def\csname PY@tok@ni\endcsname{\let\PY@bf=\textbf\def\PY@tc##1{\textcolor[rgb]{0.60,0.60,0.60}{##1}}}
\expandafter\def\csname PY@tok@na\endcsname{\def\PY@tc##1{\textcolor[rgb]{0.49,0.56,0.16}{##1}}}
\expandafter\def\csname PY@tok@nt\endcsname{\let\PY@bf=\textbf\def\PY@tc##1{\textcolor[rgb]{0.00,0.50,0.00}{##1}}}
\expandafter\def\csname PY@tok@nd\endcsname{\def\PY@tc##1{\textcolor[rgb]{0.67,0.13,1.00}{##1}}}
\expandafter\def\csname PY@tok@s\endcsname{\def\PY@tc##1{\textcolor[rgb]{0.73,0.13,0.13}{##1}}}
\expandafter\def\csname PY@tok@sd\endcsname{\let\PY@it=\textit\def\PY@tc##1{\textcolor[rgb]{0.73,0.13,0.13}{##1}}}
\expandafter\def\csname PY@tok@si\endcsname{\let\PY@bf=\textbf\def\PY@tc##1{\textcolor[rgb]{0.73,0.40,0.53}{##1}}}
\expandafter\def\csname PY@tok@se\endcsname{\let\PY@bf=\textbf\def\PY@tc##1{\textcolor[rgb]{0.73,0.40,0.13}{##1}}}
\expandafter\def\csname PY@tok@sr\endcsname{\def\PY@tc##1{\textcolor[rgb]{0.73,0.40,0.53}{##1}}}
\expandafter\def\csname PY@tok@ss\endcsname{\def\PY@tc##1{\textcolor[rgb]{0.10,0.09,0.49}{##1}}}
\expandafter\def\csname PY@tok@sx\endcsname{\def\PY@tc##1{\textcolor[rgb]{0.00,0.50,0.00}{##1}}}
\expandafter\def\csname PY@tok@m\endcsname{\def\PY@tc##1{\textcolor[rgb]{0.40,0.40,0.40}{##1}}}
\expandafter\def\csname PY@tok@gh\endcsname{\let\PY@bf=\textbf\def\PY@tc##1{\textcolor[rgb]{0.00,0.00,0.50}{##1}}}
\expandafter\def\csname PY@tok@gu\endcsname{\let\PY@bf=\textbf\def\PY@tc##1{\textcolor[rgb]{0.50,0.00,0.50}{##1}}}
\expandafter\def\csname PY@tok@gd\endcsname{\def\PY@tc##1{\textcolor[rgb]{0.63,0.00,0.00}{##1}}}
\expandafter\def\csname PY@tok@gi\endcsname{\def\PY@tc##1{\textcolor[rgb]{0.00,0.63,0.00}{##1}}}
\expandafter\def\csname PY@tok@gr\endcsname{\def\PY@tc##1{\textcolor[rgb]{1.00,0.00,0.00}{##1}}}
\expandafter\def\csname PY@tok@ge\endcsname{\let\PY@it=\textit}
\expandafter\def\csname PY@tok@gs\endcsname{\let\PY@bf=\textbf}
\expandafter\def\csname PY@tok@gp\endcsname{\let\PY@bf=\textbf\def\PY@tc##1{\textcolor[rgb]{0.00,0.00,0.50}{##1}}}
\expandafter\def\csname PY@tok@go\endcsname{\def\PY@tc##1{\textcolor[rgb]{0.53,0.53,0.53}{##1}}}
\expandafter\def\csname PY@tok@gt\endcsname{\def\PY@tc##1{\textcolor[rgb]{0.00,0.27,0.87}{##1}}}
\expandafter\def\csname PY@tok@err\endcsname{\def\PY@bc##1{\setlength{\fboxsep}{0pt}\fcolorbox[rgb]{1.00,0.00,0.00}{1,1,1}{\strut ##1}}}
\expandafter\def\csname PY@tok@kc\endcsname{\let\PY@bf=\textbf\def\PY@tc##1{\textcolor[rgb]{0.00,0.50,0.00}{##1}}}
\expandafter\def\csname PY@tok@kd\endcsname{\let\PY@bf=\textbf\def\PY@tc##1{\textcolor[rgb]{0.00,0.50,0.00}{##1}}}
\expandafter\def\csname PY@tok@kn\endcsname{\let\PY@bf=\textbf\def\PY@tc##1{\textcolor[rgb]{0.00,0.50,0.00}{##1}}}
\expandafter\def\csname PY@tok@kr\endcsname{\let\PY@bf=\textbf\def\PY@tc##1{\textcolor[rgb]{0.00,0.50,0.00}{##1}}}
\expandafter\def\csname PY@tok@bp\endcsname{\def\PY@tc##1{\textcolor[rgb]{0.00,0.50,0.00}{##1}}}
\expandafter\def\csname PY@tok@fm\endcsname{\def\PY@tc##1{\textcolor[rgb]{0.00,0.00,1.00}{##1}}}
\expandafter\def\csname PY@tok@vc\endcsname{\def\PY@tc##1{\textcolor[rgb]{0.10,0.09,0.49}{##1}}}
\expandafter\def\csname PY@tok@vg\endcsname{\def\PY@tc##1{\textcolor[rgb]{0.10,0.09,0.49}{##1}}}
\expandafter\def\csname PY@tok@vi\endcsname{\def\PY@tc##1{\textcolor[rgb]{0.10,0.09,0.49}{##1}}}
\expandafter\def\csname PY@tok@vm\endcsname{\def\PY@tc##1{\textcolor[rgb]{0.10,0.09,0.49}{##1}}}
\expandafter\def\csname PY@tok@sa\endcsname{\def\PY@tc##1{\textcolor[rgb]{0.73,0.13,0.13}{##1}}}
\expandafter\def\csname PY@tok@sb\endcsname{\def\PY@tc##1{\textcolor[rgb]{0.73,0.13,0.13}{##1}}}
\expandafter\def\csname PY@tok@sc\endcsname{\def\PY@tc##1{\textcolor[rgb]{0.73,0.13,0.13}{##1}}}
\expandafter\def\csname PY@tok@dl\endcsname{\def\PY@tc##1{\textcolor[rgb]{0.73,0.13,0.13}{##1}}}
\expandafter\def\csname PY@tok@s2\endcsname{\def\PY@tc##1{\textcolor[rgb]{0.73,0.13,0.13}{##1}}}
\expandafter\def\csname PY@tok@sh\endcsname{\def\PY@tc##1{\textcolor[rgb]{0.73,0.13,0.13}{##1}}}
\expandafter\def\csname PY@tok@s1\endcsname{\def\PY@tc##1{\textcolor[rgb]{0.73,0.13,0.13}{##1}}}
\expandafter\def\csname PY@tok@mb\endcsname{\def\PY@tc##1{\textcolor[rgb]{0.40,0.40,0.40}{##1}}}
\expandafter\def\csname PY@tok@mf\endcsname{\def\PY@tc##1{\textcolor[rgb]{0.40,0.40,0.40}{##1}}}
\expandafter\def\csname PY@tok@mh\endcsname{\def\PY@tc##1{\textcolor[rgb]{0.40,0.40,0.40}{##1}}}
\expandafter\def\csname PY@tok@mi\endcsname{\def\PY@tc##1{\textcolor[rgb]{0.40,0.40,0.40}{##1}}}
\expandafter\def\csname PY@tok@il\endcsname{\def\PY@tc##1{\textcolor[rgb]{0.40,0.40,0.40}{##1}}}
\expandafter\def\csname PY@tok@mo\endcsname{\def\PY@tc##1{\textcolor[rgb]{0.40,0.40,0.40}{##1}}}
\expandafter\def\csname PY@tok@ch\endcsname{\let\PY@it=\textit\def\PY@tc##1{\textcolor[rgb]{0.25,0.50,0.50}{##1}}}
\expandafter\def\csname PY@tok@cm\endcsname{\let\PY@it=\textit\def\PY@tc##1{\textcolor[rgb]{0.25,0.50,0.50}{##1}}}
\expandafter\def\csname PY@tok@cpf\endcsname{\let\PY@it=\textit\def\PY@tc##1{\textcolor[rgb]{0.25,0.50,0.50}{##1}}}
\expandafter\def\csname PY@tok@c1\endcsname{\let\PY@it=\textit\def\PY@tc##1{\textcolor[rgb]{0.25,0.50,0.50}{##1}}}
\expandafter\def\csname PY@tok@cs\endcsname{\let\PY@it=\textit\def\PY@tc##1{\textcolor[rgb]{0.25,0.50,0.50}{##1}}}

\def\PYZbs{\char`\\}
\def\PYZus{\char`\_}
\def\PYZob{\char`\{}
\def\PYZcb{\char`\}}
\def\PYZca{\char`\^}
\def\PYZam{\char`\&}
\def\PYZlt{\char`\<}
\def\PYZgt{\char`\>}
\def\PYZsh{\char`\#}
\def\PYZpc{\char`\%}
\def\PYZdl{\char`\$}
\def\PYZhy{\char`\-}
\def\PYZsq{\char`\'}
\def\PYZdq{\char`\"}
\def\PYZti{\char`\~}
% for compatibility with earlier versions
\def\PYZat{@}
\def\PYZlb{[}
\def\PYZrb{]}
\makeatother


    % For linebreaks inside Verbatim environment from package fancyvrb.
    \makeatletter
        \newbox\Wrappedcontinuationbox
        \newbox\Wrappedvisiblespacebox
        \newcommand*\Wrappedvisiblespace {\textcolor{red}{\textvisiblespace}}
        \newcommand*\Wrappedcontinuationsymbol {\textcolor{red}{\llap{\tiny$\m@th\hookrightarrow$}}}
        \newcommand*\Wrappedcontinuationindent {3ex }
        \newcommand*\Wrappedafterbreak {\kern\Wrappedcontinuationindent\copy\Wrappedcontinuationbox}
        % Take advantage of the already applied Pygments mark-up to insert
        % potential linebreaks for TeX processing.
        %        {, <, #, %, $, ' and ": go to next line.
        %        _, }, ^, &, >, - and ~: stay at end of broken line.
        % Use of \textquotesingle for straight quote.
        \newcommand*\Wrappedbreaksatspecials {%
            \def\PYGZus{\discretionary{\char`\_}{\Wrappedafterbreak}{\char`\_}}%
            \def\PYGZob{\discretionary{}{\Wrappedafterbreak\char`\{}{\char`\{}}%
            \def\PYGZcb{\discretionary{\char`\}}{\Wrappedafterbreak}{\char`\}}}%
            \def\PYGZca{\discretionary{\char`\^}{\Wrappedafterbreak}{\char`\^}}%
            \def\PYGZam{\discretionary{\char`\&}{\Wrappedafterbreak}{\char`\&}}%
            \def\PYGZlt{\discretionary{}{\Wrappedafterbreak\char`\<}{\char`\<}}%
            \def\PYGZgt{\discretionary{\char`\>}{\Wrappedafterbreak}{\char`\>}}%
            \def\PYGZsh{\discretionary{}{\Wrappedafterbreak\char`\#}{\char`\#}}%
            \def\PYGZpc{\discretionary{}{\Wrappedafterbreak\char`\%}{\char`\%}}%
            \def\PYGZdl{\discretionary{}{\Wrappedafterbreak\char`\$}{\char`\$}}%
            \def\PYGZhy{\discretionary{\char`\-}{\Wrappedafterbreak}{\char`\-}}%
            \def\PYGZsq{\discretionary{}{\Wrappedafterbreak\textquotesingle}{\textquotesingle}}%
            \def\PYGZdq{\discretionary{}{\Wrappedafterbreak\char`\"}{\char`\"}}%
            \def\PYGZti{\discretionary{\char`\~}{\Wrappedafterbreak}{\char`\~}}%
        }
        % Some characters . , ; ? ! / are not pygmentized.
        % This macro makes them "active" and they will insert potential linebreaks
        \newcommand*\Wrappedbreaksatpunct {%
            \lccode`\~`\.\lowercase{\def~}{\discretionary{\hbox{\char`\.}}{\Wrappedafterbreak}{\hbox{\char`\.}}}%
            \lccode`\~`\,\lowercase{\def~}{\discretionary{\hbox{\char`\,}}{\Wrappedafterbreak}{\hbox{\char`\,}}}%
            \lccode`\~`\;\lowercase{\def~}{\discretionary{\hbox{\char`\;}}{\Wrappedafterbreak}{\hbox{\char`\;}}}%
            \lccode`\~`\:\lowercase{\def~}{\discretionary{\hbox{\char`\:}}{\Wrappedafterbreak}{\hbox{\char`\:}}}%
            \lccode`\~`\?\lowercase{\def~}{\discretionary{\hbox{\char`\?}}{\Wrappedafterbreak}{\hbox{\char`\?}}}%
            \lccode`\~`\!\lowercase{\def~}{\discretionary{\hbox{\char`\!}}{\Wrappedafterbreak}{\hbox{\char`\!}}}%
            \lccode`\~`\/\lowercase{\def~}{\discretionary{\hbox{\char`\/}}{\Wrappedafterbreak}{\hbox{\char`\/}}}%
            \catcode`\.\active
            \catcode`\,\active
            \catcode`\;\active
            \catcode`\:\active
            \catcode`\?\active
            \catcode`\!\active
            \catcode`\/\active
            \lccode`\~`\~
        }
    \makeatother

    \let\OriginalVerbatim=\Verbatim
    \makeatletter
    \renewcommand{\Verbatim}[1][1]{%
        %\parskip\z@skip
        \sbox\Wrappedcontinuationbox {\Wrappedcontinuationsymbol}%
        \sbox\Wrappedvisiblespacebox {\FV@SetupFont\Wrappedvisiblespace}%
        \def\FancyVerbFormatLine ##1{\hsize\linewidth
            \vtop{\raggedright\hyphenpenalty\z@\exhyphenpenalty\z@
                \doublehyphendemerits\z@\finalhyphendemerits\z@
                \strut ##1\strut}%
        }%
        % If the linebreak is at a space, the latter will be displayed as visible
        % space at end of first line, and a continuation symbol starts next line.
        % Stretch/shrink are however usually zero for typewriter font.
        \def\FV@Space {%
            \nobreak\hskip\z@ plus\fontdimen3\font minus\fontdimen4\font
            \discretionary{\copy\Wrappedvisiblespacebox}{\Wrappedafterbreak}
            {\kern\fontdimen2\font}%
        }%

        % Allow breaks at special characters using \PYG... macros.
        \Wrappedbreaksatspecials
        % Breaks at punctuation characters . , ; ? ! and / need catcode=\active
        \OriginalVerbatim[#1,codes*=\Wrappedbreaksatpunct]%
    }
    \makeatother

    % Exact colors from NB
    \definecolor{incolor}{HTML}{303F9F}
    \definecolor{outcolor}{HTML}{D84315}
    \definecolor{cellborder}{HTML}{CFCFCF}
    \definecolor{cellbackground}{HTML}{F7F7F7}

    % prompt
    \makeatletter
    \newcommand{\boxspacing}{\kern\kvtcb@left@rule\kern\kvtcb@boxsep}
    \makeatother
    \newcommand{\prompt}[4]{
        \ttfamily\llap{{\color{#2}[#3]:\hspace{3pt}#4}}\vspace{-\baselineskip}
    }



    % Prevent overflowing lines due to hard-to-break entities
    \sloppy
    % Setup hyperref package
    \hypersetup{
      breaklinks=true,  % so long urls are correctly broken across lines
      colorlinks=true,
      urlcolor=urlcolor,
      linkcolor=linkcolor,
      citecolor=citecolor,
      }
    % Slightly bigger margins than the latex defaults

    \geometry{verbose,tmargin=1in,bmargin=1in,lmargin=1in,rmargin=1in}
\begin{document}

\nocite{*} % this command forces all references in template.bib to be printed in the bibliography

\title{Simulate Asset Price Evolutions and Reprice Risky up-and-out Call Option}

\author{
  Ng, Joe Hoong\\
  \texttt{ng\_joehoong@hotmail.com}
  \and
  Nguyen, Dang Duy Nghia \\
  \texttt{nghia002@e.ntu.edu.sg}
   \and
  Ansari, Zain Us Sami Ahmed \\
  \texttt{zainussami@gmail.com}
   \and
   Thorne, Dylan \\
  \texttt{dylan.thorne@gmail.com}
  }

\date{May. 10, 2020} % if this is omitted, the current date is used for the title page
\maketitle

\noindent
\textbf{Keywords:} European Options, Non-constant interest rate,  Local volatility, LIBOR forward rates, Zero-coupon bond and Stochastic Volatility.



% the following creates an abstract -- it can be omitted
% an example of an environment: these have the form \begin{name} ... \end{name}
\begin{abstract}
This report presents the results of the simulation of a European up-and-out call option over twelve months, but the difference between this submission and submission 1 is the implementation of a non-constant interest rate and local volatility.  We use LIBOR forward rate model to simulate interest rates.  We also use discount factor to value the option without default risk and then the value of the option with counterparty default risk.

\end{abstract}

\section{Introduction
}

In this paper, we go beyond the constant risk-free continuously-compounded rate.  There are several models to implement stochastic interest rates.  The short rate models which describe instantaneous continuously-compounded interest rate at time $t , r_t$ include the Vasicek model introduced in \cite{Vas1}, Mamon in \cite{Mamon1} advances this model by presenting approaches in obtaining the closed-form solution using the Vasicek model. Under the Vasicek model the short rate dynamics is given by,\\

\[
d r_t= \alpha (b - r_t) dt + \sigma dW_t
\]

Where $W_t$ is Brownian motion, $b$ is the level to which the short rate will tend in the long run, $\alpha$ is the rate at which the short rate will tend towards $b$ and $\sigma$ is the volatility of the short rate.  The stochastic differential equation can be solved to give $r_t$

\[
r_t= e^{-\alpha t} [r_0 + b(e^{\alpha t} - 1)+\int^t_0 \sigma e^{\alpha s} dW_s ]
\]
\\

Cox, Ingersoll and Ross also presented a stochastic differential equation for short term rates in \cite{CIR1} given by
\[
d r_t= \alpha (b - r_t) dt + \sigma \sqrt{r_t} dW_t
\]
This model prevents the short from becoming 0 but offers no closed form solution\\

The Hull-White model described in \cite{HW1} also presents stochastic differential equation for the short term rate,

\[
d r_t= (\theta(t) - \alpha(t) r_t) dt + \sigma (t) dW_t
\]
where $\theta(t) = \alpha b$ which is a non constant term, allowing the mean reversion level to vary. \\

The model we implement in this assignment is the LIBOR forward rate model to simulate interest rates.  The initial values for the LIBOR forward rates need to be calibrated to the market forward rates which can be deduced through the market zero-coupon bond prices. This continuously compounded interest rate is given by,

\[
e^{r_{ti} (t_{i+1} - t_i)}= 1 + L(t_i,t_{i+1})(t_{i+1} -t_i )
\]

Most of code implemented in this submission is derived from Module 6 \cite{M6} and Module 7 \cite{M7} of the course.

We initialize most variables as given by the question.

\begin{itemize}
    \item Option maturity is one year
    \item The option is struck at-the-money
    \item The current share price is \$100
    \item The up-and-out barrier for the option is \$150
    \item The risk-free continuously compounded interest rate is 8\%
    \item The volatility for the underlying share is 30\%
    \item The volatility for counterparty's firm value is 25\%
    \item The counterparty's debt, due in one year, is  \$175
    \item The current firm value for the counterparty is \$200
    \item The correlation between the counterparty and the stock is constant at 0.2
    \item The recovery rate with the counterparty is 25\%

\end{itemize}

\section{LIBOR Forward Rates, Stock Paths, and Counterparty Firm Values
}

In this part, we use a sample size of 100000, jointly simulate LIBOR forward rates, stock paths, and counterparty firm values. We assume that the counterparty firm and stock values are uncorrelated with LIBOR forward rates.

\subsection{Calibrate LIBOR forward rate model from zero coupon bond prices
}

    We initialize the given zero-coupon bond prices:

    \begin{tcolorbox}[breakable, size=fbox, boxrule=1pt, pad at break*=1mm,colback=cellbackground, colframe=cellborder]
\prompt{In}{incolor}{4}{\boxspacing}
\begin{Verbatim}[commandchars=\\\{\}]
\PY{n}{t} \PY{o}{=} \PY{n}{np}\PY{o}{.}\PY{n}{linspace}\PY{p}{(}\PY{l+m+mi}{0}\PY{p}{,}\PY{l+m+mi}{1}\PY{p}{,}\PY{l+m+mi}{13}\PY{p}{)}

\PY{n}{market\PYZus{}zcb\PYZus{}prices} \PY{o}{=} \PY{n}{np}\PY{o}{.}\PY{n}{array}\PY{p}{(}\PY{p}{[}\PY{l+m+mf}{1.0}\PY{p}{,} \PY{l+m+mf}{0.9938}\PY{p}{,} \PY{l+m+mf}{0.9876}\PY{p}{,} \PY{l+m+mf}{0.9815}\PY{p}{,} \PY{l+m+mf}{0.9754}\PY{p}{,} \PY{l+m+mf}{0.9694}\PY{p}{,} \PY{l+m+mf}{0.9634}\PY{p}{,} \PY{l+m+mf}{0.9574}\PY{p}{,} \PY{l+m+mf}{0.9516}\PY{p}{,}
       \PY{l+m+mf}{0.9457}\PY{p}{,} \PY{l+m+mf}{0.9399}\PY{p}{,} \PY{l+m+mf}{0.9342}\PY{p}{,} \PY{l+m+mf}{0.9285}\PY{p}{]}\PY{p}{)}
\end{Verbatim}
\end{tcolorbox}

    We next create functions to calculate the simulated bond prices from the
Vasicek model (as well as helper functions A and D). We also define
function F which is the differences between the bond prices calculated
by our model and actual market zero-coupon bond prices:

    \begin{tcolorbox}[breakable, size=fbox, boxrule=1pt, pad at break*=1mm,colback=cellbackground, colframe=cellborder]
\prompt{In}{incolor}{5}{\boxspacing}
\begin{Verbatim}[commandchars=\\\{\}]
\PY{k}{def} \PY{n+nf}{A}\PY{p}{(}\PY{n}{t1}\PY{p}{,} \PY{n}{t2}\PY{p}{,} \PY{n}{alpha}\PY{p}{)}\PY{p}{:}
    \PY{k}{return} \PY{p}{(}\PY{l+m+mi}{1}\PY{o}{\PYZhy{}}\PY{n}{np}\PY{o}{.}\PY{n}{exp}\PY{p}{(}\PY{o}{\PYZhy{}}\PY{n}{alpha}\PY{o}{*}\PY{p}{(}\PY{n}{t2}\PY{o}{\PYZhy{}}\PY{n}{t1}\PY{p}{)}\PY{p}{)}\PY{p}{)}\PY{o}{/}\PY{n}{alpha}
\PY{k}{def} \PY{n+nf}{D}\PY{p}{(}\PY{n}{t1}\PY{p}{,} \PY{n}{t2}\PY{p}{,} \PY{n}{alpha}\PY{p}{,} \PY{n}{b}\PY{p}{,} \PY{n}{sigma}\PY{p}{)}\PY{p}{:}
    \PY{n}{val1} \PY{o}{=} \PY{p}{(}\PY{n}{t2}\PY{o}{\PYZhy{}}\PY{n}{t1}\PY{o}{\PYZhy{}}\PY{n}{A}\PY{p}{(}\PY{n}{t1}\PY{p}{,}\PY{n}{t2}\PY{p}{,}\PY{n}{alpha}\PY{p}{)}\PY{p}{)}\PY{o}{*}\PY{p}{(}\PY{n}{sigma}\PY{o}{*}\PY{o}{*}\PY{l+m+mi}{2}\PY{o}{/}\PY{p}{(}\PY{l+m+mi}{2}\PY{o}{*}\PY{n}{alpha}\PY{o}{*}\PY{o}{*}\PY{l+m+mi}{2}\PY{p}{)}\PY{o}{\PYZhy{}}\PY{n}{b}\PY{p}{)}
    \PY{n}{val2} \PY{o}{=} \PY{n}{sigma}\PY{o}{*}\PY{o}{*}\PY{l+m+mi}{2}\PY{o}{*}\PY{n}{A}\PY{p}{(}\PY{n}{t1}\PY{p}{,}\PY{n}{t2}\PY{p}{,}\PY{n}{alpha}\PY{p}{)}\PY{o}{*}\PY{o}{*}\PY{l+m+mi}{2}\PY{o}{/}\PY{p}{(}\PY{l+m+mi}{4}\PY{o}{*}\PY{n}{alpha}\PY{p}{)}
    \PY{k}{return} \PY{n}{val1}\PY{o}{\PYZhy{}}\PY{n}{val2}

\PY{k}{def} \PY{n+nf}{bond\PYZus{}price\PYZus{}fun}\PY{p}{(}\PY{n}{r}\PY{p}{,}\PY{n}{t}\PY{p}{,}\PY{n}{T}\PY{p}{,} \PY{n}{alpha}\PY{p}{,} \PY{n}{b}\PY{p}{,} \PY{n}{sigma}\PY{p}{)}\PY{p}{:}
    \PY{k}{return} \PY{n}{np}\PY{o}{.}\PY{n}{exp}\PY{p}{(}\PY{o}{\PYZhy{}}\PY{n}{A}\PY{p}{(}\PY{n}{t}\PY{p}{,}\PY{n}{T}\PY{p}{,}\PY{n}{alpha}\PY{p}{)}\PY{o}{*}\PY{n}{r}\PY{o}{+}\PY{n}{D}\PY{p}{(}\PY{n}{t}\PY{p}{,}\PY{n}{T}\PY{p}{,}\PY{n}{alpha}\PY{p}{,}\PY{n}{b}\PY{p}{,}\PY{n}{sigma}\PY{p}{)}\PY{p}{)}


\PY{k}{def} \PY{n+nf}{F}\PY{p}{(}\PY{n}{x}\PY{p}{)}\PY{p}{:}
    \PY{n}{alpha} \PY{o}{=} \PY{n}{x}\PY{p}{[}\PY{l+m+mi}{0}\PY{p}{]}
    \PY{n}{b} \PY{o}{=} \PY{n}{x}\PY{p}{[}\PY{l+m+mi}{1}\PY{p}{]}
    \PY{n}{sigma} \PY{o}{=} \PY{n}{x}\PY{p}{[}\PY{l+m+mi}{2}\PY{p}{]}
    \PY{n}{r0} \PY{o}{=} \PY{n}{x}\PY{p}{[}\PY{l+m+mi}{3}\PY{p}{]}
    \PY{k}{return} \PY{n+nb}{sum}\PY{p}{(}\PY{n}{np}\PY{o}{.}\PY{n}{abs}\PY{p}{(}\PY{n}{bond\PYZus{}price\PYZus{}fun}\PY{p}{(}\PY{n}{r0}\PY{p}{,}\PY{l+m+mi}{0}\PY{p}{,}\PY{n}{t}\PY{p}{,}\PY{n}{alpha}\PY{p}{,}\PY{n}{b}\PY{p}{,}\PY{n}{sigma}\PY{p}{)}\PY{o}{\PYZhy{}}\PY{n}{market\PYZus{}zcb\PYZus{}prices}\PY{p}{)}\PY{p}{)}
\end{Verbatim}
\end{tcolorbox}

    We use the fmin\_slsqp function from scipy to calculate the optimal
model parameters, with a minimum value of F

    \begin{tcolorbox}[breakable, size=fbox, boxrule=1pt, pad at break*=1mm,colback=cellbackground, colframe=cellborder]
\prompt{In}{incolor}{6}{\boxspacing}
\begin{Verbatim}[commandchars=\\\{\}]
\PY{c+c1}{\PYZsh{}minimizing F}
\PY{n}{bnds} \PY{o}{=} \PY{p}{(}\PY{p}{(}\PY{l+m+mi}{0}\PY{p}{,}\PY{l+m+mi}{1}\PY{p}{)}\PY{p}{,}\PY{p}{(}\PY{l+m+mi}{0}\PY{p}{,}\PY{l+m+mf}{0.2}\PY{p}{)}\PY{p}{,}\PY{p}{(}\PY{l+m+mi}{0}\PY{p}{,}\PY{l+m+mf}{0.2}\PY{p}{)}\PY{p}{,} \PY{p}{(}\PY{l+m+mf}{0.00}\PY{p}{,}\PY{l+m+mf}{0.10}\PY{p}{)}\PY{p}{)}
\PY{n}{opt\PYZus{}val} \PY{o}{=} \PY{n}{scipy}\PY{o}{.}\PY{n}{optimize}\PY{o}{.}\PY{n}{fmin\PYZus{}slsqp}\PY{p}{(}\PY{n}{F}\PY{p}{,} \PY{p}{(}\PY{l+m+mf}{0.3}\PY{p}{,} \PY{l+m+mf}{0.05}\PY{p}{,} \PY{l+m+mf}{0.03}\PY{p}{,} \PY{l+m+mf}{0.05}\PY{p}{)}\PY{p}{,} \PY{n}{bounds}\PY{o}{=}\PY{n}{bnds}\PY{p}{)}
\PY{n}{opt\PYZus{}alpha} \PY{o}{=} \PY{n}{opt\PYZus{}val}\PY{p}{[}\PY{l+m+mi}{0}\PY{p}{]}
\PY{n}{opt\PYZus{}b} \PY{o}{=} \PY{n}{opt\PYZus{}val}\PY{p}{[}\PY{l+m+mi}{1}\PY{p}{]}
\PY{n}{opt\PYZus{}sigma} \PY{o}{=} \PY{n}{opt\PYZus{}val}\PY{p}{[}\PY{l+m+mi}{2}\PY{p}{]}
\PY{n}{opt\PYZus{}r0} \PY{o}{=} \PY{n}{opt\PYZus{}val}\PY{p}{[}\PY{l+m+mi}{3}\PY{p}{]}
\end{Verbatim}
\end{tcolorbox}

    \begin{Verbatim}[commandchars=\\\{\}]
Optimization terminated successfully.    (Exit mode 0)
            Current function value: 0.0002564991813429618
            Iterations: 10
            Function evaluations: 74
            Gradient evaluations: 10
    \end{Verbatim}

    \begin{tcolorbox}[breakable, size=fbox, boxrule=1pt, pad at break*=1mm,colback=cellbackground, colframe=cellborder]
\prompt{In}{incolor}{7}{\boxspacing}
\begin{Verbatim}[commandchars=\\\{\}]
\PY{n+nb}{print}\PY{p}{(}\PY{l+s+s2}{\PYZdq{}}\PY{l+s+s2}{Optimal alpha: }\PY{l+s+si}{\PYZob{}:.3f\PYZcb{}}\PY{l+s+s2}{\PYZdq{}}\PY{o}{.}\PY{n}{format}\PY{p}{(}\PY{n}{opt\PYZus{}val}\PY{p}{[}\PY{l+m+mi}{0}\PY{p}{]}\PY{p}{)}\PY{p}{)}
\PY{n+nb}{print}\PY{p}{(}\PY{l+s+s2}{\PYZdq{}}\PY{l+s+s2}{Optimal b: }\PY{l+s+si}{\PYZob{}:.3f\PYZcb{}}\PY{l+s+s2}{\PYZdq{}}\PY{o}{.}\PY{n}{format}\PY{p}{(}\PY{n}{opt\PYZus{}val}\PY{p}{[}\PY{l+m+mi}{1}\PY{p}{]}\PY{p}{)}\PY{p}{)}
\PY{n+nb}{print}\PY{p}{(}\PY{l+s+s2}{\PYZdq{}}\PY{l+s+s2}{Optimal sigma }\PY{l+s+si}{\PYZob{}:.3f\PYZcb{}}\PY{l+s+s2}{\PYZdq{}}\PY{o}{.}\PY{n}{format}\PY{p}{(}\PY{n}{opt\PYZus{}val}\PY{p}{[}\PY{l+m+mi}{2}\PY{p}{]}\PY{p}{)}\PY{p}{)}
\PY{n+nb}{print}\PY{p}{(}\PY{l+s+s2}{\PYZdq{}}\PY{l+s+s2}{Optimal r0: }\PY{l+s+si}{\PYZob{}:.3f\PYZcb{}}\PY{l+s+s2}{\PYZdq{}}\PY{o}{.}\PY{n}{format}\PY{p}{(}\PY{n}{opt\PYZus{}val}\PY{p}{[}\PY{l+m+mi}{3}\PY{p}{]}\PY{p}{)}\PY{p}{)}
\end{Verbatim}
\end{tcolorbox}

    \begin{Verbatim}[commandchars=\\\{\}]
Optimal alpha: 0.273
Optimal b: 0.069
Optimal sigma 0.028
Optimal r0: 0.075
    \end{Verbatim}

    We plot the actual market bond prices, with model-derived bond prices,
and they look like a close fit.

    \begin{tcolorbox}[breakable, size=fbox, boxrule=1pt, pad at break*=1mm,colback=cellbackground, colframe=cellborder]
\prompt{In}{incolor}{8}{\boxspacing}
\begin{Verbatim}[commandchars=\\\{\}]
\PY{n}{model\PYZus{}prices} \PY{o}{=} \PY{n}{bond\PYZus{}price\PYZus{}fun}\PY{p}{(}\PY{n}{opt\PYZus{}r0}\PY{p}{,}\PY{l+m+mi}{0}\PY{p}{,}\PY{n}{t}\PY{p}{,} \PY{n}{opt\PYZus{}alpha}\PY{p}{,} \PY{n}{opt\PYZus{}b}\PY{p}{,} \PY{n}{opt\PYZus{}sigma}\PY{p}{)}
\PY{n}{model\PYZus{}yield} \PY{o}{=} \PY{o}{\PYZhy{}}\PY{n}{np}\PY{o}{.}\PY{n}{log}\PY{p}{(}\PY{n}{model\PYZus{}prices}\PY{p}{)}\PY{o}{/}\PY{n}{t}

\PY{n}{plt}\PY{o}{.}\PY{n}{plot}\PY{p}{(}\PY{n}{t}\PY{p}{,}\PY{n}{market\PYZus{}zcb\PYZus{}prices}\PY{p}{,} \PY{n}{label}\PY{o}{=}\PY{l+s+s1}{\PYZsq{}}\PY{l+s+s1}{Market Prices}\PY{l+s+s1}{\PYZsq{}}\PY{p}{)}
\PY{n}{plt}\PY{o}{.}\PY{n}{plot}\PY{p}{(}\PY{n}{t}\PY{p}{,} \PY{n}{model\PYZus{}prices}\PY{p}{,} \PY{l+s+s1}{\PYZsq{}}\PY{l+s+s1}{.}\PY{l+s+s1}{\PYZsq{}}\PY{p}{,} \PY{n}{label}\PY{o}{=}\PY{l+s+s1}{\PYZsq{}}\PY{l+s+s1}{Calibrated Prices}\PY{l+s+s1}{\PYZsq{}}\PY{p}{)}
\PY{n}{plt}\PY{o}{.}\PY{n}{legend}\PY{p}{(}\PY{p}{)}
\PY{n}{plt}\PY{o}{.}\PY{n}{show}\PY{p}{(}\PY{p}{)}
\end{Verbatim}
\end{tcolorbox}


    \begin{center}
    \adjustimage{max size={0.9\linewidth}{0.9\paperheight}}{images/output_15_1.png}
    \end{center}
    { \hspace*{\fill} \\}


\subsection{Simulate LIBOR rate paths
}


    We first initialize the parameter \(\sigma_j\)

    \begin{tcolorbox}[breakable, size=fbox, boxrule=1pt, pad at break*=1mm,colback=cellbackground, colframe=cellborder]
\prompt{In}{incolor}{9}{\boxspacing}
\begin{Verbatim}[commandchars=\\\{\}]
\PY{n}{sigmaj} \PY{o}{=} \PY{l+m+mf}{0.2}
\end{Verbatim}
\end{tcolorbox}

    We use paramters we obtained above to recreate the Vasicek bond prices:

    \begin{tcolorbox}[breakable, size=fbox, boxrule=1pt, pad at break*=1mm,colback=cellbackground, colframe=cellborder]
\prompt{In}{incolor}{10}{\boxspacing}
\begin{Verbatim}[commandchars=\\\{\}]
\PY{k}{def} \PY{n+nf}{A}\PY{p}{(}\PY{n}{t1}\PY{p}{,} \PY{n}{t2}\PY{p}{)}\PY{p}{:}
    \PY{k}{return} \PY{p}{(}\PY{l+m+mi}{1}\PY{o}{\PYZhy{}}\PY{n}{np}\PY{o}{.}\PY{n}{exp}\PY{p}{(}\PY{o}{\PYZhy{}}\PY{n}{opt\PYZus{}alpha}\PY{o}{*}\PY{p}{(}\PY{n}{t2}\PY{o}{\PYZhy{}}\PY{n}{t1}\PY{p}{)}\PY{p}{)}\PY{p}{)}\PY{o}{/}\PY{n}{opt\PYZus{}alpha}

\PY{k}{def} \PY{n+nf}{C}\PY{p}{(}\PY{n}{t1}\PY{p}{,} \PY{n}{t2}\PY{p}{)}\PY{p}{:}
    \PY{n}{val1} \PY{o}{=} \PY{p}{(}\PY{n}{t2}\PY{o}{\PYZhy{}}\PY{n}{t1}\PY{o}{\PYZhy{}}\PY{n}{A}\PY{p}{(}\PY{n}{t1}\PY{p}{,}\PY{n}{t2}\PY{p}{)}\PY{p}{)}\PY{o}{*}\PY{p}{(}\PY{n}{opt\PYZus{}sigma}\PY{o}{*}\PY{o}{*}\PY{l+m+mi}{2}\PY{o}{/}\PY{p}{(}\PY{l+m+mi}{2}\PY{o}{*}\PY{n}{opt\PYZus{}alpha}\PY{o}{*}\PY{o}{*}\PY{l+m+mi}{2}\PY{p}{)}\PY{o}{\PYZhy{}}\PY{n}{opt\PYZus{}b}\PY{p}{)}
    \PY{n}{val2} \PY{o}{=} \PY{n}{opt\PYZus{}sigma}\PY{o}{*}\PY{o}{*}\PY{l+m+mi}{2}\PY{o}{*}\PY{n}{A}\PY{p}{(}\PY{n}{t1}\PY{p}{,}\PY{n}{t2}\PY{p}{)}\PY{o}{*}\PY{o}{*}\PY{l+m+mi}{2}\PY{o}{/}\PY{p}{(}\PY{l+m+mi}{4}\PY{o}{*}\PY{n}{opt\PYZus{}alpha}\PY{p}{)}
    \PY{k}{return} \PY{n}{val1} \PY{o}{\PYZhy{}} \PY{n}{val2}

\PY{k}{def} \PY{n+nf}{bond\PYZus{}price}\PY{p}{(}\PY{n}{r}\PY{p}{,}\PY{n}{t}\PY{p}{,}\PY{n}{T}\PY{p}{)}\PY{p}{:}
    \PY{k}{return} \PY{n}{np}\PY{o}{.}\PY{n}{exp}\PY{p}{(}\PY{o}{\PYZhy{}}\PY{n}{A}\PY{p}{(}\PY{n}{t}\PY{p}{,}\PY{n}{T}\PY{p}{)}\PY{o}{*}\PY{n}{r}\PY{o}{+}\PY{n}{C}\PY{p}{(}\PY{n}{t}\PY{p}{,}\PY{n}{T}\PY{p}{)}\PY{p}{)}

\PY{n}{vasi\PYZus{}bond} \PY{o}{=} \PY{n}{bond\PYZus{}price}\PY{p}{(}\PY{n}{opt\PYZus{}r0}\PY{p}{,} \PY{l+m+mi}{0}\PY{p}{,} \PY{n}{t}\PY{p}{)}
\end{Verbatim}
\end{tcolorbox}

    The prices calculated from the Vasicek model are close to the ZCB prices
given by the assignment:

    \begin{tcolorbox}[breakable, size=fbox, boxrule=1pt, pad at break*=1mm,colback=cellbackground, colframe=cellborder]
\prompt{In}{incolor}{11}{\boxspacing}
\begin{Verbatim}[commandchars=\\\{\}]
\PY{n+nb}{print}\PY{p}{(}\PY{n}{vasi\PYZus{}bond}\PY{p}{)}
\end{Verbatim}
\end{tcolorbox}

    \begin{Verbatim}[commandchars=\\\{\}]
[1.         0.99377572 0.98760087 0.98147516 0.97539831 0.96936998
 0.96338983 0.95745752 0.95157266 0.94573489 0.93994381 0.93419901
 0.9285001 ]
    \end{Verbatim}

    We now initialize the matrices we will use to store the Monte Carlo
simulations, for both basic Monte Carlo and Predictor-Corrector method.

    \begin{tcolorbox}[breakable, size=fbox, boxrule=1pt, pad at break*=1mm,colback=cellbackground, colframe=cellborder]
\prompt{In}{incolor}{12}{\boxspacing}
\begin{Verbatim}[commandchars=\\\{\}]
\PY{n}{n\PYZus{}simulations} \PY{o}{=} \PY{n}{sample\PYZus{}size}
\PY{n}{n\PYZus{}steps} \PY{o}{=} \PY{n+nb}{len}\PY{p}{(}\PY{n}{t}\PY{p}{)}

\PY{n}{mc\PYZus{}forward} \PY{o}{=} \PY{n}{np}\PY{o}{.}\PY{n}{ones}\PY{p}{(}\PY{p}{[}\PY{n}{n\PYZus{}simulations}\PY{p}{,} \PY{n}{n\PYZus{}steps}\PY{o}{\PYZhy{}}\PY{l+m+mi}{1}\PY{p}{]}\PY{p}{)}\PY{o}{*}\PY{p}{(}\PY{n}{vasi\PYZus{}bond}\PY{p}{[}\PY{p}{:}\PY{o}{\PYZhy{}}\PY{l+m+mi}{1}\PY{p}{]}\PY{o}{\PYZhy{}}\PY{n}{vasi\PYZus{}bond}\PY{p}{[}\PY{l+m+mi}{1}\PY{p}{:}\PY{p}{]}\PY{p}{)}\PY{o}{/}\PY{p}{(}\PY{n}{vasi\PYZus{}bond}\PY{p}{[}\PY{l+m+mi}{1}\PY{p}{:}\PY{p}{]}\PY{p}{)}
\PY{n}{predcorr\PYZus{}forward} \PY{o}{=} \PY{n}{np}\PY{o}{.}\PY{n}{ones}\PY{p}{(}\PY{p}{[}\PY{n}{n\PYZus{}simulations}\PY{p}{,} \PY{n}{n\PYZus{}steps}\PY{o}{\PYZhy{}}\PY{l+m+mi}{1}\PY{p}{]}\PY{p}{)}\PY{o}{*}\PY{p}{(}\PY{n}{vasi\PYZus{}bond}\PY{p}{[}\PY{p}{:}\PY{o}{\PYZhy{}}\PY{l+m+mi}{1}\PY{p}{]}\PY{o}{\PYZhy{}}\PY{n}{vasi\PYZus{}bond}\PY{p}{[}\PY{l+m+mi}{1}\PY{p}{:}\PY{p}{]}\PY{p}{)}\PY{o}{/}\PY{p}{(}\PY{n}{vasi\PYZus{}bond}\PY{p}{[}\PY{l+m+mi}{1}\PY{p}{:}\PY{p}{]}\PY{p}{)}
\PY{n}{predcorr\PYZus{}capfac} \PY{o}{=} \PY{n}{np}\PY{o}{.}\PY{n}{ones}\PY{p}{(}\PY{p}{[}\PY{n}{n\PYZus{}simulations}\PY{p}{,} \PY{n}{n\PYZus{}steps}\PY{p}{]}\PY{p}{)}
\PY{n}{mc\PYZus{}capfac} \PY{o}{=} \PY{n}{np}\PY{o}{.}\PY{n}{ones}\PY{p}{(}\PY{p}{[}\PY{n}{n\PYZus{}simulations}\PY{p}{,} \PY{n}{n\PYZus{}steps}\PY{p}{]}\PY{p}{)}

\PY{n}{delta} \PY{o}{=} \PY{n}{np}\PY{o}{.}\PY{n}{ones}\PY{p}{(}\PY{p}{[}\PY{n}{n\PYZus{}simulations}\PY{p}{,} \PY{n}{n\PYZus{}steps} \PY{o}{\PYZhy{}} \PY{l+m+mi}{1}\PY{p}{]}\PY{p}{)}\PY{o}{*}\PY{p}{(}\PY{n}{t}\PY{p}{[}\PY{l+m+mi}{1}\PY{p}{:}\PY{p}{]}\PY{o}{\PYZhy{}}\PY{n}{t}\PY{p}{[}\PY{p}{:}\PY{o}{\PYZhy{}}\PY{l+m+mi}{1}\PY{p}{]}\PY{p}{)}
\end{Verbatim}
\end{tcolorbox}

    We now run the Monte Carlo simulation for each time step:

    \begin{tcolorbox}[breakable, size=fbox, boxrule=1pt, pad at break*=1mm,colback=cellbackground, colframe=cellborder]
\prompt{In}{incolor}{13}{\boxspacing}
\begin{Verbatim}[commandchars=\\\{\}]
\PY{k}{for} \PY{n}{i} \PY{o+ow}{in} \PY{n+nb}{range}\PY{p}{(}\PY{l+m+mi}{1}\PY{p}{,} \PY{n}{n\PYZus{}steps}\PY{p}{)}\PY{p}{:}
    \PY{n}{Z} \PY{o}{=} \PY{n}{norm}\PY{o}{.}\PY{n}{rvs}\PY{p}{(}\PY{n}{size}\PY{o}{=}\PY{p}{[}\PY{n}{n\PYZus{}simulations}\PY{p}{,}\PY{l+m+mi}{1}\PY{p}{]}\PY{p}{)}
    
    \PY{n}{muhat} \PY{o}{=} \PY{n}{np}\PY{o}{.}\PY{n}{cumsum}\PY{p}{(}\PY{n}{delta}\PY{p}{[}\PY{p}{:}\PY{p}{,} \PY{n}{i}\PY{p}{:}\PY{p}{]}\PY{o}{*}\PY{n}{mc\PYZus{}forward}\PY{p}{[}\PY{p}{:}\PY{p}{,} \PY{n}{i}\PY{p}{:}\PY{p}{]}\PY{o}{*}\PY{n}{sigmaj}\PY{o}{*}\PY{o}{*}\PY{l+m+mi}{2}\PY{o}{/}\PY{p}{(}\PY{l+m+mi}{1}\PY{o}{+}\PY{n}{delta}\PY{p}{[}\PY{p}{:}\PY{p}{,} \PY{n}{i}\PY{p}{:}\PY{p}{]}\PY{o}{*}\PY{n}{mc\PYZus{}forward}\PY{p}{[}\PY{p}{:}\PY{p}{,}\PY{n}{i}\PY{p}{:}\PY{p}{]}\PY{p}{)}\PY{p}{,} \PY{n}{axis}\PY{o}{=}\PY{l+m+mi}{1}\PY{p}{)}
    \PY{n}{mc\PYZus{}forward}\PY{p}{[}\PY{p}{:}\PY{p}{,}\PY{n}{i}\PY{p}{:}\PY{p}{]} \PY{o}{=} \PY{n}{mc\PYZus{}forward}\PY{p}{[}\PY{p}{:}\PY{p}{,}\PY{n}{i}\PY{p}{:}\PY{p}{]}\PY{o}{*}\PY{n}{np}\PY{o}{.}\PY{n}{exp}\PY{p}{(}\PY{p}{(}\PY{n}{muhat}\PY{o}{\PYZhy{}}\PY{n}{sigmaj}\PY{o}{*}\PY{o}{*}\PY{l+m+mi}{2}\PY{o}{/}\PY{l+m+mi}{2}\PY{p}{)}\PY{o}{*}\PY{n}{delta}\PY{p}{[}\PY{p}{:}\PY{p}{,}\PY{n}{i}\PY{p}{:}\PY{p}{]}\PY{o}{+}\PY{n}{sigmaj}\PY{o}{*}\PY{n}{np}\PY{o}{.}\PY{n}{sqrt}\PY{p}{(}\PY{n}{delta}\PY{p}{[}\PY{p}{:}\PY{p}{,}\PY{n}{i}\PY{p}{:}\PY{p}{]}\PY{p}{)}\PY{o}{*}\PY{n}{Z}\PY{p}{)}
    
    \PY{n}{mu\PYZus{}initial} \PY{o}{=} \PY{n}{np}\PY{o}{.}\PY{n}{cumsum}\PY{p}{(}\PY{n}{delta}\PY{p}{[}\PY{p}{:}\PY{p}{,}\PY{n}{i}\PY{p}{:}\PY{p}{]}\PY{o}{*}\PY{n}{predcorr\PYZus{}forward}\PY{p}{[}\PY{p}{:}\PY{p}{,}\PY{n}{i}\PY{p}{:}\PY{p}{]}\PY{o}{*}\PY{n}{sigmaj}\PY{o}{*}\PY{o}{*}\PY{l+m+mi}{2}\PY{o}{/}\PY{p}{(}\PY{l+m+mi}{1}\PY{o}{+}\PY{n}{delta}\PY{p}{[}\PY{p}{:}\PY{p}{,}\PY{n}{i}\PY{p}{:}\PY{p}{]}\PY{o}{*}\PY{n}{predcorr\PYZus{}forward}\PY{p}{[}\PY{p}{:}\PY{p}{,}\PY{n}{i}\PY{p}{:}\PY{p}{]}\PY{p}{)}\PY{p}{,} \PY{n}{axis}\PY{o}{=}\PY{l+m+mi}{1}\PY{p}{)}
    \PY{n}{for\PYZus{}temp} \PY{o}{=} \PY{n}{predcorr\PYZus{}forward}\PY{p}{[}\PY{p}{:}\PY{p}{,}\PY{n}{i}\PY{p}{:}\PY{p}{]}\PY{o}{*}\PY{n}{np}\PY{o}{.}\PY{n}{exp}\PY{p}{(}\PY{p}{(}\PY{n}{mu\PYZus{}initial}\PY{o}{\PYZhy{}}\PY{n}{sigmaj}\PY{o}{*}\PY{o}{*}\PY{l+m+mi}{2}\PY{o}{/}\PY{l+m+mi}{2}\PY{p}{)}\PY{o}{*}\PY{n}{delta}\PY{p}{[}\PY{p}{:}\PY{p}{,}\PY{n}{i}\PY{p}{:}\PY{p}{]}\PY{o}{+}\PY{n}{sigmaj}\PY{o}{*}\PY{n}{np}\PY{o}{.}\PY{n}{sqrt}\PY{p}{(}\PY{n}{delta}\PY{p}{[}\PY{p}{:}\PY{p}{,}\PY{n}{i}\PY{p}{:}\PY{p}{]}\PY{p}{)}\PY{o}{*}\PY{n}{Z}\PY{p}{)}
    \PY{n}{mu\PYZus{}term} \PY{o}{=} \PY{n}{np}\PY{o}{.}\PY{n}{cumsum}\PY{p}{(}\PY{n}{delta}\PY{p}{[}\PY{p}{:}\PY{p}{,}\PY{n}{i}\PY{p}{:}\PY{p}{]}\PY{o}{*}\PY{n}{for\PYZus{}temp}\PY{o}{*}\PY{n}{sigmaj}\PY{o}{*}\PY{o}{*}\PY{l+m+mi}{2}\PY{o}{/}\PY{p}{(}\PY{l+m+mi}{1}\PY{o}{+}\PY{n}{delta}\PY{p}{[}\PY{p}{:}\PY{p}{,}\PY{n}{i}\PY{p}{:}\PY{p}{]}\PY{o}{*}\PY{n}{for\PYZus{}temp}\PY{p}{)}\PY{p}{,} \PY{n}{axis}\PY{o}{=}\PY{l+m+mi}{1}\PY{p}{)}
    \PY{n}{predcorr\PYZus{}forward}\PY{p}{[}\PY{p}{:}\PY{p}{,}\PY{n}{i}\PY{p}{:}\PY{p}{]} \PY{o}{=} \PY{n}{predcorr\PYZus{}forward}\PY{p}{[}\PY{p}{:}\PY{p}{,}\PY{n}{i}\PY{p}{:}\PY{p}{]}\PY{o}{*}\PY{n}{np}\PY{o}{.}\PY{n}{exp}\PY{p}{(}\PY{p}{(}\PY{n}{mu\PYZus{}initial}\PY{o}{+}\PY{n}{mu\PYZus{}term}\PY{o}{\PYZhy{}}\PY{n}{sigmaj}\PY{o}{*}\PY{o}{*}\PY{l+m+mi}{2}\PY{p}{)}\PY{o}{*}\PY{n}{delta}\PY{p}{[}\PY{p}{:}\PY{p}{,}\PY{n}{i}\PY{p}{:}\PY{p}{]}\PY{o}{/}\PY{l+m+mi}{2}\PY{o}{+}\PY{n}{sigmaj}\PY{o}{*}\PY{n}{np}\PY{o}{.}\PY{n}{sqrt}\PY{p}{(}\PY{n}{delta}\PY{p}{[}\PY{p}{:}\PY{p}{,}\PY{n}{i}\PY{p}{:}\PY{p}{]}\PY{p}{)}\PY{o}{*}\PY{n}{Z}\PY{p}{)}
\end{Verbatim}
\end{tcolorbox}

    From our Monte Carlo simulation, we now calculate the capitalization
factors and bond prices, and plot them to compare them with the Vasicek
bond prices.

    \begin{tcolorbox}[breakable, size=fbox, boxrule=1pt, pad at break*=1mm,colback=cellbackground, colframe=cellborder]
\prompt{In}{incolor}{14}{\boxspacing}
\begin{Verbatim}[commandchars=\\\{\}]
\PY{n}{mc\PYZus{}capfac}\PY{p}{[}\PY{p}{:}\PY{p}{,}\PY{l+m+mi}{1}\PY{p}{:}\PY{p}{]} \PY{o}{=} \PY{n}{np}\PY{o}{.}\PY{n}{cumprod}\PY{p}{(}\PY{l+m+mi}{1}\PY{o}{+}\PY{n}{mc\PYZus{}forward}\PY{p}{,} \PY{n}{axis}\PY{o}{=}\PY{l+m+mi}{1}\PY{p}{)}
\PY{n}{predcorr\PYZus{}capfac}\PY{p}{[}\PY{p}{:}\PY{p}{,}\PY{l+m+mi}{1}\PY{p}{:}\PY{p}{]} \PY{o}{=} \PY{n}{np}\PY{o}{.}\PY{n}{cumprod}\PY{p}{(}\PY{l+m+mi}{1}\PY{o}{+}\PY{n}{predcorr\PYZus{}forward}\PY{p}{,} \PY{n}{axis}\PY{o}{=}\PY{l+m+mi}{1}\PY{p}{)}

\PY{n}{mc\PYZus{}price} \PY{o}{=} \PY{n}{mc\PYZus{}capfac}\PY{o}{*}\PY{o}{*}\PY{p}{(}\PY{o}{\PYZhy{}}\PY{l+m+mi}{1}\PY{p}{)}
\PY{n}{predcorr\PYZus{}price} \PY{o}{=} \PY{n}{predcorr\PYZus{}capfac}\PY{o}{*}\PY{o}{*}\PY{p}{(}\PY{o}{\PYZhy{}}\PY{l+m+mi}{1}\PY{p}{)}

\PY{n}{mc\PYZus{}final} \PY{o}{=} \PY{n}{np}\PY{o}{.}\PY{n}{mean}\PY{p}{(}\PY{n}{mc\PYZus{}price}\PY{p}{,} \PY{n}{axis}\PY{o}{=}\PY{l+m+mi}{0}\PY{p}{)}
\PY{n}{predcorr\PYZus{}final} \PY{o}{=} \PY{n}{np}\PY{o}{.}\PY{n}{mean}\PY{p}{(}\PY{n}{predcorr\PYZus{}price}\PY{p}{,} \PY{n}{axis}\PY{o}{=}\PY{l+m+mi}{0}\PY{p}{)}
\end{Verbatim}
\end{tcolorbox}

    \begin{tcolorbox}[breakable, size=fbox, boxrule=1pt, pad at break*=1mm,colback=cellbackground, colframe=cellborder]
\prompt{In}{incolor}{15}{\boxspacing}
\begin{Verbatim}[commandchars=\\\{\}]
\PY{n}{plt}\PY{o}{.}\PY{n}{xlabel}\PY{p}{(}\PY{l+s+s2}{\PYZdq{}}\PY{l+s+s2}{Maturity}\PY{l+s+s2}{\PYZdq{}}\PY{p}{)}
\PY{n}{plt}\PY{o}{.}\PY{n}{ylabel}\PY{p}{(}\PY{l+s+s2}{\PYZdq{}}\PY{l+s+s2}{Bond Price}\PY{l+s+s2}{\PYZdq{}}\PY{p}{)}
\PY{n}{plt}\PY{o}{.}\PY{n}{plot}\PY{p}{(}\PY{n}{t}\PY{p}{,}\PY{n}{vasi\PYZus{}bond}\PY{p}{,} \PY{n}{label}\PY{o}{=}\PY{l+s+s2}{\PYZdq{}}\PY{l+s+s2}{Vasicek Bond Prices}\PY{l+s+s2}{\PYZdq{}}\PY{p}{)}

\PY{n}{plt}\PY{o}{.}\PY{n}{plot}\PY{p}{(}\PY{n}{t}\PY{p}{,} \PY{n}{mc\PYZus{}final}\PY{p}{,} \PY{l+s+s1}{\PYZsq{}}\PY{l+s+s1}{o}\PY{l+s+s1}{\PYZsq{}}\PY{p}{,} \PY{n}{label}\PY{o}{=}\PY{l+s+s2}{\PYZdq{}}\PY{l+s+s2}{Simple Monte Carlo Bond Prices}\PY{l+s+s2}{\PYZdq{}}\PY{p}{)}
\PY{n}{plt}\PY{o}{.}\PY{n}{plot}\PY{p}{(}\PY{n}{t}\PY{p}{,} \PY{n}{predcorr\PYZus{}final}\PY{p}{,} \PY{l+s+s1}{\PYZsq{}}\PY{l+s+s1}{x}\PY{l+s+s1}{\PYZsq{}}\PY{p}{,} \PY{n}{label}\PY{o}{=}\PY{l+s+s2}{\PYZdq{}}\PY{l+s+s2}{Predictor\PYZhy{}Corrector Bond Prices}\PY{l+s+s2}{\PYZdq{}}\PY{p}{)}
\PY{n}{plt}\PY{o}{.}\PY{n}{legend}\PY{p}{(}\PY{p}{)}
\PY{n}{plt}\PY{o}{.}\PY{n}{show}\PY{p}{(}\PY{p}{)}
\end{Verbatim}
\end{tcolorbox}

    \begin{center}
    \adjustimage{max size={0.9\linewidth}{0.9\paperheight}}{images/output_29_0.png}
    \end{center}
    { \hspace*{\fill} \\}
    
    From our simulation of forward rates (we take the Predictor-Corrector
method, we use the formula
\(e^{r_{t_i} (t_{i+1}-t_i)} = 1 + L(t_i,t_{i+1})(t_{i+1}-t_i)\) to
obtain the continuous compounded interest rates:

    \begin{tcolorbox}[breakable, size=fbox, boxrule=1pt, pad at break*=1mm,colback=cellbackground, colframe=cellborder]
\prompt{In}{incolor}{16}{\boxspacing}
\begin{Verbatim}[commandchars=\\\{\}]
\PY{n}{r\PYZus{}sim} \PY{o}{=} \PY{n}{np}\PY{o}{.}\PY{n}{log}\PY{p}{(}\PY{l+m+mi}{1} \PY{o}{+} \PY{n}{predcorr\PYZus{}forward}\PY{o}{*}\PY{p}{(}\PY{n}{delta}\PY{p}{)}\PY{p}{)}\PY{o}{/}\PY{n}{delta}
\end{Verbatim}
\end{tcolorbox}

    We also calculate an annualized form of the interest rates:

    \begin{tcolorbox}[breakable, size=fbox, boxrule=1pt, pad at break*=1mm,colback=cellbackground, colframe=cellborder]
\prompt{In}{incolor}{17}{\boxspacing}
\begin{Verbatim}[commandchars=\\\{\}]
\PY{n}{r\PYZus{}sim\PYZus{}annualized} \PY{o}{=} \PY{n}{pd}\PY{o}{.}\PY{n}{DataFrame}\PY{p}{(}\PY{n}{r\PYZus{}sim}\PY{o}{/}\PY{n}{delta}\PY{p}{)}
\end{Verbatim}
\end{tcolorbox}

    \begin{tcolorbox}[breakable, size=fbox, boxrule=1pt, pad at break*=1mm,colback=cellbackground, colframe=cellborder]
\prompt{In}{incolor}{18}{\boxspacing}
\begin{Verbatim}[commandchars=\\\{\}]
\PY{n}{r\PYZus{}sim\PYZus{}annualized}
\end{Verbatim}
\end{tcolorbox}

            \begin{tcolorbox}[breakable, size=fbox, boxrule=.5pt, pad at break*=1mm, opacityfill=0]
\prompt{Out}{outcolor}{18}{\boxspacing}
\begin{Verbatim}[commandchars=\\\{\}]
              0         1         2         3         4         5         6  \textbackslash{}
0       0.07514  0.066032  0.073336  0.075253  0.065992  0.066157  0.066794
1       0.07514  0.072083  0.076464  0.081468  0.079424  0.082719  0.085049
2       0.07514  0.078051  0.079510  0.077283  0.080866  0.078631  0.085738
3       0.07514  0.067973  0.066249  0.068189  0.061341  0.061618  0.058422
4       0.07514  0.068828  0.067538  0.068650  0.064811  0.064151  0.067009
{\ldots}         {\ldots}       {\ldots}       {\ldots}       {\ldots}       {\ldots}       {\ldots}       {\ldots}
999995  0.07514  0.080051  0.081460  0.080036  0.078720  0.085394  0.090455
999996  0.07514  0.075648  0.076774  0.074305  0.081762  0.077945  0.075159
999997  0.07514  0.080705  0.077654  0.076033  0.068839  0.066677  0.062133
999998  0.07514  0.085071  0.085031  0.082831  0.083125  0.087771  0.077675
999999  0.07514  0.079983  0.080474  0.078617  0.073621  0.071960  0.072492

               7         8         9        10        11
0       0.071269  0.076366  0.074026  0.080224  0.076966
1       0.085452  0.094079  0.106641  0.120261  0.116014
2       0.082785  0.072790  0.076300  0.066587  0.063343
3       0.061475  0.058621  0.057793  0.058849  0.054900
4       0.063756  0.062794  0.065755  0.066556  0.072365
{\ldots}          {\ldots}       {\ldots}       {\ldots}       {\ldots}       {\ldots}
999995  0.091226  0.086187  0.086976  0.081274  0.075437
999996  0.079487  0.069895  0.067660  0.062574  0.059106
999997  0.059384  0.056089  0.055484  0.056012  0.054232
999998  0.075775  0.075541  0.073908  0.078261  0.073692
999999  0.068380  0.066112  0.071977  0.071854  0.065560

[1000000 rows x 12 columns]
\end{Verbatim}
\end{tcolorbox}
        
\subsection{Generate stock and firm values
}

    Similar to the first groupwork assignment, we use a Cholesky
decomposition to generate the correlated price paths

    \begin{tcolorbox}[breakable, size=fbox, boxrule=1pt, pad at break*=1mm,colback=cellbackground, colframe=cellborder]
\prompt{In}{incolor}{19}{\boxspacing}
\begin{Verbatim}[commandchars=\\\{\}]
\PY{k}{def} \PY{n+nf}{next\PYZus{}share\PYZus{}price}\PY{p}{(}\PY{n}{prev\PYZus{}price}\PY{p}{,} \PY{n}{r}\PY{p}{,} \PY{n}{dT}\PY{p}{,} \PY{n}{sigma\PYZus{}const}\PY{p}{,} \PY{n}{gamma}\PY{p}{,} \PY{n}{sample\PYZus{}size}\PY{p}{,} \PY{n}{Z}\PY{p}{,} \PY{n}{varying\PYZus{}vol} \PY{o}{=} \PY{k+kc}{True}\PY{p}{)}\PY{p}{:}

    \PY{k}{if} \PY{n}{varying\PYZus{}vol}\PY{p}{:}
        \PY{n}{sigma} \PY{o}{=} \PY{n}{sigma\PYZus{}const}\PY{o}{*}\PY{p}{(}\PY{n}{prev\PYZus{}price}\PY{p}{)}\PY{o}{*}\PY{o}{*}\PY{p}{(}\PY{n}{gamma}\PY{o}{\PYZhy{}}\PY{l+m+mi}{1}\PY{p}{)}
    \PY{k}{else}\PY{p}{:}
        \PY{n}{sigma} \PY{o}{=} \PY{n}{sigma\PYZus{}const}\PY{o}{*}\PY{p}{(}\PY{n}{S0}\PY{p}{)}\PY{o}{*}\PY{o}{*}\PY{p}{(}\PY{n}{gamma}\PY{o}{\PYZhy{}}\PY{l+m+mi}{1}\PY{p}{)}
    
    \PY{k}{return} \PY{n}{prev\PYZus{}price}\PY{o}{*}\PY{n}{np}\PY{o}{.}\PY{n}{exp}\PY{p}{(}\PY{n}{np}\PY{o}{.}\PY{n}{cumsum}\PY{p}{(}\PY{p}{(}\PY{n}{r}\PY{o}{\PYZhy{}}\PY{p}{(}\PY{n}{sigma}\PY{o}{*}\PY{o}{*}\PY{l+m+mi}{2}\PY{p}{)}\PY{o}{/}\PY{l+m+mi}{2}\PY{p}{)}\PY{o}{*}\PY{p}{(}\PY{n}{dT}\PY{p}{)}\PY{o}{+}\PY{p}{(}\PY{n}{sigma}\PY{p}{)}\PY{o}{*}\PY{p}{(}\PY{n}{np}\PY{o}{.}\PY{n}{sqrt}\PY{p}{(}\PY{n}{dT}\PY{p}{)}\PY{p}{)}\PY{o}{*}\PY{n}{Z}\PY{p}{,}\PY{l+m+mi}{1}\PY{p}{)}\PY{p}{)}

\PY{k}{def} \PY{n+nf}{generate\PYZus{}share\PYZus{}and\PYZus{}firm\PYZus{}price}\PY{p}{(}\PY{n}{S0}\PY{p}{,} \PY{n}{v\PYZus{}0}\PY{p}{,} \PY{n}{r\PYZus{}sim}\PY{p}{,} \PY{n}{sigma\PYZus{}const}\PY{p}{,} \PY{n}{gamma}\PY{p}{,} \PY{n}{corr}\PY{p}{,} \PY{n}{T}\PY{p}{,} \PY{n}{sample\PYZus{}size}\PY{p}{,} \PY{n}{timesteps} \PY{o}{=} \PY{l+m+mi}{12}\PY{p}{)}\PY{p}{:}
    \PY{n}{corr\PYZus{}matrix} \PY{o}{=} \PY{n}{np}\PY{o}{.}\PY{n}{array}\PY{p}{(}\PY{p}{[}\PY{p}{[}\PY{l+m+mi}{1}\PY{p}{,} \PY{n}{corr}\PY{p}{]}\PY{p}{,} \PY{p}{[}\PY{n}{corr}\PY{p}{,} \PY{l+m+mi}{1}\PY{p}{]}\PY{p}{]}\PY{p}{)}
    \PY{n}{norm\PYZus{}matrix} \PY{o}{=} \PY{n}{stats}\PY{o}{.}\PY{n}{norm}\PY{o}{.}\PY{n}{rvs}\PY{p}{(}\PY{n}{size} \PY{o}{=} \PY{n}{np}\PY{o}{.}\PY{n}{array}\PY{p}{(}\PY{p}{[}\PY{n}{sample\PYZus{}size}\PY{p}{,} \PY{l+m+mi}{2}\PY{p}{,} \PY{n}{timesteps}\PY{p}{]}\PY{p}{)}\PY{p}{)}
    \PY{n}{corr\PYZus{}norm\PYZus{}matrix} \PY{o}{=} \PY{n}{np}\PY{o}{.}\PY{n}{matmul}\PY{p}{(}\PY{n}{np}\PY{o}{.}\PY{n}{linalg}\PY{o}{.}\PY{n}{cholesky}\PY{p}{(}\PY{n}{corr\PYZus{}matrix}\PY{p}{)}\PY{p}{,} \PY{n}{norm\PYZus{}matrix}\PY{p}{)}
    
    
    \PY{n}{share\PYZus{}price\PYZus{}path} \PY{o}{=} \PY{n}{pd}\PY{o}{.}\PY{n}{DataFrame}\PY{p}{(}\PY{n}{next\PYZus{}share\PYZus{}price}\PY{p}{(}\PY{n}{S0}\PY{p}{,} \PY{n}{r\PYZus{}sim}\PY{p}{,} \PY{l+m+mi}{1}\PY{o}{/}\PY{n}{timesteps}\PY{p}{,} \PY{n}{sigma\PYZus{}const}\PY{p}{,} \PY{n}{gamma}\PY{p}{,} \PY{n}{sample\PYZus{}size}\PY{p}{,} \PY{n}{Z}\PY{o}{=}\PY{n}{corr\PYZus{}norm\PYZus{}matrix}\PY{p}{[}\PY{p}{:}\PY{p}{,}\PY{l+m+mi}{0}\PY{p}{,}\PY{p}{]}\PY{p}{)}\PY{p}{)}
    \PY{n}{share\PYZus{}price\PYZus{}path} \PY{o}{=} \PY{n}{share\PYZus{}price\PYZus{}path}\PY{o}{.}\PY{n}{transpose}\PY{p}{(}\PY{p}{)}
    
    \PY{n}{first\PYZus{}row} \PY{o}{=} \PY{n}{pd}\PY{o}{.}\PY{n}{DataFrame}\PY{p}{(}\PY{p}{[}\PY{n}{S0}\PY{p}{]}\PY{o}{*}\PY{n}{sample\PYZus{}size}\PY{p}{)}
    \PY{n}{first\PYZus{}row} \PY{o}{=} \PY{n}{first\PYZus{}row}\PY{o}{.}\PY{n}{transpose}\PY{p}{(}\PY{p}{)}
    \PY{n}{share\PYZus{}price\PYZus{}path} \PY{o}{=} \PY{n}{pd}\PY{o}{.}\PY{n}{concat}\PY{p}{(}\PY{p}{[}\PY{n}{first\PYZus{}row}\PY{p}{,} \PY{n}{share\PYZus{}price\PYZus{}path}\PY{p}{]}\PY{p}{)}
    \PY{n}{share\PYZus{}price\PYZus{}path} \PY{o}{=} \PY{n}{share\PYZus{}price\PYZus{}path}\PY{o}{.}\PY{n}{reset\PYZus{}index}\PY{p}{(}\PY{n}{drop}\PY{o}{=}\PY{k+kc}{True}\PY{p}{)}

    \PY{n}{firm\PYZus{}price\PYZus{}path} \PY{o}{=} \PY{n}{pd}\PY{o}{.}\PY{n}{DataFrame}\PY{p}{(}\PY{n}{next\PYZus{}share\PYZus{}price}\PY{p}{(}\PY{n}{v\PYZus{}0}\PY{p}{,} \PY{n}{r\PYZus{}sim}\PY{p}{,} \PY{l+m+mi}{1}\PY{o}{/}\PY{n}{timesteps}\PY{p}{,} \PY{n}{sigma\PYZus{}const}\PY{p}{,} \PY{n}{gamma}\PY{p}{,} \PY{n}{sample\PYZus{}size}\PY{p}{,} \PY{n}{Z}\PY{o}{=}\PY{n}{corr\PYZus{}norm\PYZus{}matrix}\PY{p}{[}\PY{p}{:}\PY{p}{,}\PY{l+m+mi}{1}\PY{p}{,}\PY{p}{]}\PY{p}{)}\PY{p}{)}
    \PY{n}{firm\PYZus{}price\PYZus{}path} \PY{o}{=} \PY{n}{firm\PYZus{}price\PYZus{}path}\PY{o}{.}\PY{n}{transpose}\PY{p}{(}\PY{p}{)}
    
    \PY{n}{first\PYZus{}row} \PY{o}{=} \PY{n}{pd}\PY{o}{.}\PY{n}{DataFrame}\PY{p}{(}\PY{p}{[}\PY{n}{v\PYZus{}0}\PY{p}{]}\PY{o}{*}\PY{n}{sample\PYZus{}size}\PY{p}{)}
    \PY{n}{first\PYZus{}row} \PY{o}{=} \PY{n}{first\PYZus{}row}\PY{o}{.}\PY{n}{transpose}\PY{p}{(}\PY{p}{)}
    \PY{n}{firm\PYZus{}price\PYZus{}path} \PY{o}{=} \PY{n}{pd}\PY{o}{.}\PY{n}{concat}\PY{p}{(}\PY{p}{[}\PY{n}{first\PYZus{}row}\PY{p}{,} \PY{n}{firm\PYZus{}price\PYZus{}path}\PY{p}{]}\PY{p}{)}
    \PY{n}{firm\PYZus{}price\PYZus{}path} \PY{o}{=} \PY{n}{firm\PYZus{}price\PYZus{}path}\PY{o}{.}\PY{n}{reset\PYZus{}index}\PY{p}{(}\PY{n}{drop}\PY{o}{=}\PY{k+kc}{True}\PY{p}{)}

    \PY{k}{return} \PY{p}{[}\PY{n}{share\PYZus{}price\PYZus{}path}\PY{p}{,}\PY{n}{firm\PYZus{}price\PYZus{}path}\PY{p}{]}  
\end{Verbatim}
\end{tcolorbox}

    \begin{tcolorbox}[breakable, size=fbox, boxrule=1pt, pad at break*=1mm,colback=cellbackground, colframe=cellborder]
\prompt{In}{incolor}{20}{\boxspacing}
\begin{Verbatim}[commandchars=\\\{\}]
\PY{n}{share\PYZus{}prices}\PY{p}{,} \PY{n}{firm\PYZus{}prices} \PY{o}{=} \PY{n}{generate\PYZus{}share\PYZus{}and\PYZus{}firm\PYZus{}price}\PY{p}{(}\PY{n}{S0}\PY{p}{,} \PY{n}{v\PYZus{}0}\PY{p}{,} \PY{n}{r\PYZus{}sim\PYZus{}annualized}\PY{p}{,} \PY{n}{sigma\PYZus{}const}\PY{p}{,} \PY{n}{gamma}\PY{p}{,} \PY{n}{corr}\PY{p}{,} \PY{n}{T}\PY{p}{,} \PY{n}{sample\PYZus{}size}\PY{p}{,} \PY{n}{timesteps} \PY{o}{=} \PY{l+m+mi}{12}\PY{p}{)}
\end{Verbatim}
\end{tcolorbox}

We also plot the first 1000 stock price and firm value paths simulated:

    \begin{tcolorbox}[breakable, size=fbox, boxrule=1pt, pad at break*=1mm,colback=cellbackground, colframe=cellborder]
\prompt{In}{incolor}{23}{\boxspacing}
\begin{Verbatim}[commandchars=\\\{\}]
\PY{n}{share\PYZus{}prices}\PY{o}{.}\PY{n}{iloc}\PY{p}{[}\PY{p}{:}\PY{p}{,}\PY{l+m+mi}{0}\PY{p}{:}\PY{l+m+mi}{1000}\PY{p}{]}\PY{o}{.}\PY{n}{plot}\PY{p}{(}\PY{n}{title}\PY{o}{=}\PY{l+s+s1}{\PYZsq{}}\PY{l+s+s1}{Share price over 12 months}\PY{l+s+s1}{\PYZsq{}}\PY{p}{,} \PY{n}{legend}\PY{o}{=}\PY{k+kc}{False}\PY{p}{)}\PY{p}{;}
\end{Verbatim}
\end{tcolorbox}

    \begin{center}
    \adjustimage{max size={0.9\linewidth}{0.9\paperheight}}{images/output_43_0.png}
    \end{center}
    { \hspace*{\fill} \\}
    
    \begin{tcolorbox}[breakable, size=fbox, boxrule=1pt, pad at break*=1mm,colback=cellbackground, colframe=cellborder]
\prompt{In}{incolor}{24}{\boxspacing}
\begin{Verbatim}[commandchars=\\\{\}]
\PY{n}{firm\PYZus{}prices}\PY{o}{.}\PY{n}{iloc}\PY{p}{[}\PY{p}{:}\PY{p}{,}\PY{l+m+mi}{0}\PY{p}{:}\PY{l+m+mi}{1000}\PY{p}{]}\PY{o}{.}\PY{n}{plot}\PY{p}{(}\PY{n}{title}\PY{o}{=}\PY{l+s+s1}{\PYZsq{}}\PY{l+s+s1}{Firm price over 12 months}\PY{l+s+s1}{\PYZsq{}}\PY{p}{,} \PY{n}{legend}\PY{o}{=}\PY{k+kc}{False}\PY{p}{)}\PY{p}{;}
\end{Verbatim}
\end{tcolorbox}

    \begin{center}
    \adjustimage{max size={0.9\linewidth}{0.9\paperheight}}{images/output_44_0.png}
    \end{center}
    { \hspace*{\fill} \\}


\section{Discount Factor and Value of the Up-and-Out Call Option
}

In this section we use the capitalisation factor calculated in the section aboove to  calculate the one-year discount factor which applies for each simulation, and use this to find first the value of the option for the jointly simulated stock and firm paths with no default risk, and then the value of the option with counterparty default risk. \\

We first calculated the one year discount factor, by inverting the capitalisation factor. The capitalisation factor is calculated by taking the cumulative product of the interest rate between each timestep.

    \begin{tcolorbox}[breakable, size=fbox, boxrule=1pt, pad at break*=1mm,colback=cellbackground, colframe=cellborder]
\prompt{In}{incolor}{25}{\boxspacing}
\begin{Verbatim}[commandchars=\\\{\}]
\PY{n}{one\PYZus{}year\PYZus{}disc\PYZus{}fac} \PY{o}{=} \PY{l+m+mi}{1}\PY{o}{/}\PY{n}{np}\PY{o}{.}\PY{n}{cumprod}\PY{p}{(}\PY{l+m+mi}{1}\PY{o}{+}\PY{n}{r\PYZus{}sim}\PY{p}{,}\PY{l+m+mi}{1}\PY{p}{)}\PY{p}{[}\PY{p}{:}\PY{p}{,}\PY{o}{\PYZhy{}}\PY{l+m+mi}{1}\PY{p}{]}
\PY{n}{one\PYZus{}year\PYZus{}disc\PYZus{}fac}
\end{Verbatim}
\end{tcolorbox}

            \begin{tcolorbox}[breakable, size=fbox, boxrule=.5pt, pad at break*=1mm, opacityfill=0]
\prompt{Out}{outcolor}{25}{\boxspacing}
\begin{Verbatim}[commandchars=\\\{\}]
array([0.93045792, 0.91464088, 0.92665426, {\ldots}, 0.93658426, 0.92383486,
       0.92979433])
\end{Verbatim}
\end{tcolorbox}

Next, we calculate the Default-Free Option Value. One difference is that we multiple the payoff by the one year discount factor, instead of multiplying with  $e^{rT}$
        
    \begin{tcolorbox}[breakable, size=fbox, boxrule=1pt, pad at break*=1mm,colback=cellbackground, colframe=cellborder]
\prompt{In}{incolor}{26}{\boxspacing}
\begin{Verbatim}[commandchars=\\\{\}]
\PY{c+c1}{\PYZsh{} define payoff for up\PYZhy{}and\PYZhy{}out call option}
\PY{k}{def} \PY{n+nf}{payoff}\PY{p}{(}\PY{n}{S\PYZus{}t}\PY{p}{,} \PY{n}{K}\PY{p}{,} \PY{n}{L}\PY{p}{)}\PY{p}{:}
    \PY{n}{stopped\PYZus{}S} \PY{o}{=} \PY{n}{S\PYZus{}t}\PY{o}{.}\PY{n}{iloc}\PY{p}{[}\PY{o}{\PYZhy{}}\PY{l+m+mi}{1}\PY{p}{]}\PY{o}{.}\PY{n}{where}\PY{p}{(}\PY{p}{(}\PY{n}{S\PYZus{}t} \PY{o}{\PYZlt{}} \PY{n}{L}\PY{p}{)}\PY{o}{.}\PY{n}{all}\PY{p}{(}\PY{p}{)}\PY{p}{,} \PY{l+m+mi}{0}\PY{p}{)}
    \PY{k}{return} \PY{n}{np}\PY{o}{.}\PY{n}{maximum}\PY{p}{(}\PY{n}{stopped\PYZus{}S} \PY{o}{\PYZhy{}} \PY{n}{K}\PY{p}{,} \PY{l+m+mi}{0}\PY{p}{)}\PY{o}{.}\PY{n}{to\PYZus{}numpy}\PY{p}{(}\PY{p}{)}
\end{Verbatim}
\end{tcolorbox}

    \begin{tcolorbox}[breakable, size=fbox, boxrule=1pt, pad at break*=1mm,colback=cellbackground, colframe=cellborder]
\prompt{In}{incolor}{27}{\boxspacing}
\begin{Verbatim}[commandchars=\\\{\}]
\PY{c+c1}{\PYZsh{} Estimate the default\PYZhy{}free value of the option:}
\PY{n}{option\PYZus{}estimate} \PY{o}{=} \PY{p}{[}\PY{p}{]}
\PY{n}{option\PYZus{}std} \PY{o}{=} \PY{p}{[}\PY{p}{]}


\PY{n}{payoffs} \PY{o}{=} \PY{n}{payoff}\PY{p}{(}\PY{n}{share\PYZus{}prices}\PY{p}{,} \PY{n}{K}\PY{p}{,} \PY{n}{L}\PY{p}{)}
\PY{n}{option\PYZus{}price} \PY{o}{=} \PY{n}{one\PYZus{}year\PYZus{}disc\PYZus{}fac}\PY{o}{*}\PY{n}{payoffs}
\PY{n}{option\PYZus{}estimate} \PY{o}{=} \PY{n}{option\PYZus{}price}\PY{o}{.}\PY{n}{mean}\PY{p}{(}\PY{p}{)}
\PY{n}{option\PYZus{}std} \PY{o}{=} \PY{n}{option\PYZus{}price}\PY{o}{.}\PY{n}{std}\PY{p}{(}\PY{p}{)}\PY{o}{/}\PY{n}{np}\PY{o}{.}\PY{n}{sqrt}\PY{p}{(}\PY{n}{sample\PYZus{}size}\PY{p}{)}
\end{Verbatim}
\end{tcolorbox}

    \begin{tcolorbox}[breakable, size=fbox, boxrule=1pt, pad at break*=1mm,colback=cellbackground, colframe=cellborder]
\prompt{In}{incolor}{28}{\boxspacing}
\begin{Verbatim}[commandchars=\\\{\}]
\PY{n+nb}{print}\PY{p}{(}\PY{l+s+s2}{\PYZdq{}}\PY{l+s+s2}{Default\PYZhy{}free option price }\PY{l+s+si}{\PYZob{}:.3f\PYZcb{}}\PY{l+s+s2}{\PYZdq{}}\PY{o}{.}\PY{n}{format}\PY{p}{(}\PY{n}{option\PYZus{}estimate}\PY{p}{)}\PY{p}{)}
\PY{n+nb}{print}\PY{p}{(}\PY{l+s+s2}{\PYZdq{}}\PY{l+s+s2}{Default\PYZhy{}free option price standard deviation }\PY{l+s+si}{\PYZob{}:.3f\PYZcb{}}\PY{l+s+s2}{\PYZdq{}}\PY{o}{.}\PY{n}{format}\PY{p}{(}\PY{n}{option\PYZus{}std}\PY{p}{)}\PY{p}{)}
\end{Verbatim}
\end{tcolorbox}

    \begin{Verbatim}[commandchars=\\\{\}]
Default-free option price 8.296
Default-free option price standard deviation 0.008
    \end{Verbatim}
    
Next, we incorporate the CVA Adjustment similar to the first submission.    

    \begin{tcolorbox}[breakable, size=fbox, boxrule=1pt, pad at break*=1mm,colback=cellbackground, colframe=cellborder]
\prompt{In}{incolor}{29}{\boxspacing}
\begin{Verbatim}[commandchars=\\\{\}]
\PY{n}{payoffs} \PY{o}{=} \PY{n}{payoff}\PY{p}{(}\PY{n}{share\PYZus{}prices}\PY{p}{,} \PY{n}{K}\PY{p}{,} \PY{n}{L}\PY{p}{)}
\PY{n}{term\PYZus{}firm\PYZus{}vals} \PY{o}{=} \PY{n}{firm\PYZus{}prices}\PY{o}{.}\PY{n}{iloc}\PY{p}{[}\PY{o}{\PYZhy{}}\PY{l+m+mi}{1}\PY{p}{]}\PY{o}{.}\PY{n}{to\PYZus{}numpy}\PY{p}{(}\PY{p}{)}
\PY{n}{amount\PYZus{}lost} \PY{o}{=} \PY{n}{one\PYZus{}year\PYZus{}disc\PYZus{}fac}\PY{o}{*}\PY{p}{(}\PY{l+m+mi}{1}\PY{o}{\PYZhy{}}\PY{n}{recovery\PYZus{}rate}\PY{p}{)}\PY{o}{*}\PY{p}{(}\PY{n}{term\PYZus{}firm\PYZus{}vals} \PY{o}{\PYZlt{}} \PY{n}{debt}\PY{p}{)}\PY{o}{*}\PY{n}{payoffs}
\PY{n}{cva\PYZus{}estimate} \PY{o}{=} \PY{n}{amount\PYZus{}lost}\PY{o}{.}\PY{n}{mean}\PY{p}{(}\PY{p}{)}
\PY{n}{cva\PYZus{}std} \PY{o}{=} \PY{n}{amount\PYZus{}lost}\PY{o}{.}\PY{n}{std}\PY{p}{(}\PY{p}{)}\PY{o}{/}\PY{n}{np}\PY{o}{.}\PY{n}{sqrt}\PY{p}{(}\PY{n}{sample\PYZus{}size}\PY{p}{)}

\PY{n}{option\PYZus{}cva\PYZus{}price} \PY{o}{=} \PY{n}{option\PYZus{}price} \PY{o}{\PYZhy{}} \PY{n}{amount\PYZus{}lost}
\PY{n}{option\PYZus{}cva\PYZus{}adjusted\PYZus{}prices} \PY{o}{=} \PY{n}{option\PYZus{}cva\PYZus{}price}\PY{o}{.}\PY{n}{mean}\PY{p}{(}\PY{p}{)}
\PY{n}{option\PYZus{}cva\PYZus{}adjusted\PYZus{}std} \PY{o}{=} \PY{n}{option\PYZus{}cva\PYZus{}price}\PY{o}{.}\PY{n}{std}\PY{p}{(}\PY{p}{)}\PY{o}{/}\PY{n}{np}\PY{o}{.}\PY{n}{sqrt}\PY{p}{(}\PY{n}{sample\PYZus{}size}\PY{p}{)}
\end{Verbatim}
\end{tcolorbox}

    \begin{tcolorbox}[breakable, size=fbox, boxrule=1pt, pad at break*=1mm,colback=cellbackground, colframe=cellborder]
\prompt{In}{incolor}{30}{\boxspacing}
\begin{Verbatim}[commandchars=\\\{\}]
\PY{n+nb}{print}\PY{p}{(}\PY{l+s+s2}{\PYZdq{}}\PY{l+s+s2}{Credit value adjustment }\PY{l+s+si}{\PYZob{}:.3f\PYZcb{}}\PY{l+s+s2}{\PYZdq{}}\PY{o}{.}\PY{n}{format}\PY{p}{(}\PY{n}{cva\PYZus{}estimate}\PY{p}{)}\PY{p}{)}
\PY{n+nb}{print}\PY{p}{(}\PY{l+s+s2}{\PYZdq{}}\PY{l+s+s2}{Credit value adjustment standard deviation }\PY{l+s+si}{\PYZob{}:.3f\PYZcb{}}\PY{l+s+s2}{\PYZdq{}}\PY{o}{.}\PY{n}{format}\PY{p}{(}\PY{n}{cva\PYZus{}std}\PY{p}{)}\PY{p}{)}

\PY{n+nb}{print}\PY{p}{(}\PY{l+s+s2}{\PYZdq{}}\PY{l+s+s2}{CVA\PYZhy{}adjusted option price }\PY{l+s+si}{\PYZob{}:.3f\PYZcb{}}\PY{l+s+s2}{\PYZdq{}}\PY{o}{.}\PY{n}{format}\PY{p}{(}\PY{n}{option\PYZus{}cva\PYZus{}adjusted\PYZus{}prices}\PY{p}{)}\PY{p}{)}
\PY{n+nb}{print}\PY{p}{(}\PY{l+s+s2}{\PYZdq{}}\PY{l+s+s2}{CVA\PYZhy{}adjusted option price standard deviation }\PY{l+s+si}{\PYZob{}:.3f\PYZcb{}}\PY{l+s+s2}{\PYZdq{}}\PY{o}{.}\PY{n}{format}\PY{p}{(}\PY{n}{option\PYZus{}cva\PYZus{}adjusted\PYZus{}std}\PY{p}{)}\PY{p}{)}
\end{Verbatim}
\end{tcolorbox}

    \begin{Verbatim}[commandchars=\\\{\}]
Credit value adjustment 0.018
Credit value adjustment standard deviation 0.000
CVA-adjusted option price 8.278
CVA-adjusted option price standard deviation 0.008
    \end{Verbatim}




\section{Conclusion}
In this paper, we simulated correlated firm and share price paths. From this, we priced an up-and-out call option at 8.299 with a default-free risk profile, and at 8.281 for the CVA-adjusted price. Considering that the same parameters are used as the first submission, where the Black-Scholes-Merton model was used to arrive at a price of 5.697, we observe that both the default risk, the variable interest rate and local volatility have all added a premium to the option price. These conditions, which are more aligned to observed market conditions, gave a total increase of 45\% over the previously calculated price, with the default risk accounting for only 0.2\% of the difference. These results underline the importance of choosing the correct model and performing accurate calibration in order to calculate instrument prices.


\newpage
\section*{} \label{bibsection}


% the second parameter MMMMM should be as long as the longest label you use, in my case Smoller -- if you use % numbers only, use 99
% use \cite{refname} to refer to bibliography item \bibitem{refname}
% LaTeX assigns a number, unless you use \bibitem[Name]{refname} -- in this case
% LaTeX prints Name when you use \cite{refname}
\begin{thebibliography}{MMMMM}
\bibitem{Vas1} Vasicek, O. (1977). An equilibrium characterization of the term structure, Journal of financial economics 5(2): 177-188.
\bibitem{Mamon1} Mamon, R. S. (2004). Three ways to solve for bond prices in the vasicek model, Advances in Decision Sciences 8(1): 1-14.
\bibitem{HW1} Hull, J. and White, A. (2001). The general hull-white model and supercalibration, Financial Analysts Journal pp. 34-43.
\bibitem{CIR1} Cox, J. C., Ingersoll Jr, J. E. and Ross, S. A. (1985). An intertemporal general equilibrium model of asset prices, Econometrica: Journal of the Econometric Society pp. 363-384.
\bibitem{AMST1} Albrecher, H., Mayer, P., Schoutens, W. and Tistaert, J. (2007).
``The Little Heston Trap'', Wilmott (1): 83--92.
\bibitem{Cox1}  Cox, John. "Notes on option pricing I: Constant elasticity of variance diffusions." Unpublished note, Stanford University, Graduate School of Business (1975).
\bibitem{M5} MScFE630 Computational Finance Module 5: Monte Carlo Methods for Risk Management
\bibitem{M6} MScFE630 Computational Finance Module 6: Pricing Interest Rate Options
\bibitem{M7} MScFE630 Computational Finance Module 7: Calibration

\end{thebibliography}

\end{document}
